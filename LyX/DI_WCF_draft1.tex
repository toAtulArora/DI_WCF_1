%% LyX 2.3.6.1 created this file.  For more info, see http://www.lyx.org/.
%% Do not edit unless you really know what you are doing.
\documentclass[british]{article}
\usepackage{amsmath}
\usepackage{amsthm}
\usepackage{libertineRoman}
\usepackage{biolinum}
\renewcommand{\ttdefault}{lmtt}
\usepackage[libertine]{newtxmath}
\usepackage[T1]{fontenc}
\usepackage[latin9]{inputenc}
\usepackage{geometry}
\geometry{verbose,tmargin=1in,bmargin=1in,lmargin=1in,rmargin=1in,headheight=1in,headsep=1in,footskip=0.7in}
\usepackage{color}
\usepackage{refstyle}
\usepackage{booktabs}
\usepackage{graphicx}
\usepackage{wasysym}

\makeatletter

%%%%%%%%%%%%%%%%%%%%%%%%%%%%%% LyX specific LaTeX commands.

\AtBeginDocument{\providecommand\Eqref[1]{\ref{Eq:#1}}}
\AtBeginDocument{\providecommand\Defref[1]{\ref{Def:#1}}}
\AtBeginDocument{\providecommand\Claimref[1]{\ref{Claim:#1}}}
\AtBeginDocument{\providecommand\Algref[1]{\ref{Alg:#1}}}
\AtBeginDocument{\providecommand\Subsecref[1]{\ref{Subsec:#1}}}
\AtBeginDocument{\providecommand\Secref[1]{\ref{Sec:#1}}}
\AtBeginDocument{\providecommand\Subsubsecref[1]{\ref{Subsubsec:#1}}}
\AtBeginDocument{\providecommand\Lemref[1]{\ref{Lem:#1}}}
\AtBeginDocument{\providecommand\Figref[1]{\ref{Fig:#1}}}
\AtBeginDocument{\providecommand\Assuref[1]{\ref{Assu:#1}}}
\AtBeginDocument{\providecommand\Remref[1]{\ref{Rem:#1}}}
\AtBeginDocument{\providecommand\Exaref[1]{\ref{Exa:#1}}}
\AtBeginDocument{\providecommand\Corref[1]{\ref{Cor:#1}}}
\newcommand{\noun}[1]{\textsc{#1}}
%% Because html converters don't know tabularnewline
\providecommand{\tabularnewline}{\\}
\RS@ifundefined{subsecref}
  {\newref{subsec}{name = \RSsectxt}}
  {}
\RS@ifundefined{thmref}
  {\def\RSthmtxt{theorem~}\newref{thm}{name = \RSthmtxt}}
  {}
\RS@ifundefined{lemref}
  {\def\RSlemtxt{lemma~}\newref{lem}{name = \RSlemtxt}}
  {}


%%%%%%%%%%%%%%%%%%%%%%%%%%%%%% Textclass specific LaTeX commands.
\theoremstyle{plain}
\newtheorem{thm}{\protect\theoremname}
\theoremstyle{definition}
\newtheorem{defn}[thm]{\protect\definitionname}
\theoremstyle{plain}
\newtheorem{assumption}[thm]{\protect\assumptionname}
\theoremstyle{remark}
\newtheorem{claim}[thm]{\protect\claimname}
\theoremstyle{plain}
\newtheorem{lyxalgorithm}[thm]{\protect\algorithmname}
\theoremstyle{plain}
\newtheorem{lem}[thm]{\protect\lemmaname}
\theoremstyle{remark}
\newtheorem{rem}[thm]{\protect\remarkname}
\theoremstyle{plain}
\newtheorem{conjecture}[thm]{\protect\conjecturename}
\theoremstyle{definition}
\newtheorem{example}[thm]{\protect\examplename}
\theoremstyle{plain}
\newtheorem{cor}[thm]{\protect\corollaryname}
\theoremstyle{plain}
\newtheorem{prop}[thm]{\protect\propositionname}

%%%%%%%%%%%%%%%%%%%%%%%%%%%%%% User specified LaTeX commands.
\usepackage{color}
\definecolor{purple}{RGB}{120,20,120}
\newcommand\branchcolor[2]{{\color{#1} #2}}
\newcommand\branchpurple[1]{{\color{purple} #1}}

\usepackage{hyperref}

\hypersetup{colorlinks=true,urlcolor=blue}



\newref{thm}{name=theorem~,Name=Theorem~,names=theorems~,Names=Theorems~}
\newref{def}{name=definition~,Name=Definition~,names=definitions~,Names=Definitions~}
\newref{alg}{name=algorithm~,Name=Algorithm~,names=algorithms~,Names=Algorithms~}
%\newref{cor}{name=corollary~,Name=Corollary~,names=corollaries~,Names=Corollaries~}
\newref{lem}{name=lemma~,Name=Lemma~,names=lemmas~,Names=Lemmas~}
\newref{claim}{name=claim~,Name=Claim~,names=claims~,Names=Claims~}
\newref{sec}{name=section~,Name=Section~,names=sections~,Names=Sections~}
\newref{subsec}{name=section~,Name=Section~,names=sections~,Names=Sections~}
\newref{subsubsec}{name=section~,Name=Section~,names=sections~,Names=Sections~}
%\newref{prop}{name=proposition~,Name=Proposition~,names=propositions~,Names=Propositions~}
%\newref{conj}{name=conjecture~,Name=Conjecture~,names=conjectures~,Names=Conjectures~}
\newref{assu}{name=assumption~,Name=Assumption~,names=assumptions~,Names=Assumptions~}
\newref{rem}{name=remark~,Name=Remark~,names=remarks~,Names=Remarks~}
%\newref{alg}{name=algorithm~,Name=Algorithm~,names=algorithms~,Names=Algorithms~}
\newref{fact}{name=fact~,Name=Fact~,names=facts~,Names=Facts~}
\newref{exa}{name=example~,Name=Example~,names=examples~,Names=Examples~}

\makeatother

\usepackage{babel}
\providecommand{\algorithmname}{Algorithm}
\providecommand{\assumptionname}{Assumption}
\providecommand{\claimname}{Claim}
\providecommand{\conjecturename}{Conjecture}
\providecommand{\corollaryname}{Corollary}
\providecommand{\definitionname}{Definition}
\providecommand{\examplename}{Example}
\providecommand{\lemmaname}{Lemma}
\providecommand{\propositionname}{Proposition}
\providecommand{\remarkname}{Remark}
\providecommand{\theoremname}{Theorem}

\begin{document}
\title{Improved Device Independent Weak Coin Flipping Protocols}
\maketitle
\begin{abstract}
{[}OUTDATED: Needs to be rewritten{]}\\
We report a device independent weak coin flipping protocol\footnote{which are analysed }
with $P_{A}^{*}\le\cos^{2}(\pi/8)$ and $P_{B}^{*}\le0.667...$, by
making seemingly minor changes to the best known protocol due to SCAKPM'11
{[}10.1103/PhysRevLett.106.220501{]}, with $P_{A}^{*}\le\cos^{2}(\pi/8)\approx0.85$
and $P_{B}^{*}\le3/4=0.75$. In terms of bias, we improve the SCAKPM'11
result from $\approx0.336$ to $\approx0.3199$. This improvement
is due to two ingredients: a self-testing (of GHZ) step and an extra
cheat detection step for Bob. We also introduce a new bias suppression
technique that ekes out further security from the abort probability
to obtain ... Note that the SCAKPM'11 result held for both strong
and weak coin flipping; ours holds only for the latter. TODO: Fix
me!
\end{abstract}
%
\global\long\def\tr{\text{tr}}%

\tableofcontents{}

\section{Introduction}

INTERNAL/Atul: Colour coding---Purple is for informal discussions,
black is for formal statements and blue is for proofs. We can remove
these from the final version; I put it to minimise verbiage.

\subsection{About Weak Coin Flipping}

\branchcolor{purple}{Secure two-party computation is a cryptographic setting where two
parties, conventionally called Alice and Bob, receive inputs $x$
and $y$ and their goal is to compute some function $f_{A}(x,y)$
and $f_{B}(x,y)$ respectively which depends on both their inputs.
However, they do not wish to reveal their inputs. Coin flipping (CF)
is a cryptographic primitive in this setting, i.e. a building block
for constructing more applicable secure two-party cryptographic schemes,
where Alice and Bob wish to exchange messages and agree on a random
bit, without trusting each other. A protocol that implements coin
flipping must protect an honest player from a malicious\footnote{(or cheating, we use these adjectives interchangeably)}
player. 

A weaker primitive, unsurprisingly, known as \emph{weak coin flipping}
(WCF) is where a zero corresponds to Alice winning and one corresponds
to Bob winning. It is weaker because now the protocol has to protect
Alice from a malicious Bob who tries to bias the outcome towards one
(and not towards zero) and conversely, it must protect Bob from a
malicious Alice who tries to bias the outcome towards zero (and not
towards one). To emphasise the distinction, the former primitive is
often termed \emph{strong coin flipping} (SCF).

We primarily focus on WCF in this article and begin with introducing
some notation. We denote by $P_{A}^{*}$ the highest probability of
a malicious Alice convincing an honest Bob that she won (i.e. in the
WCF protocol, Alice uses her best cheating strategy against Bob who
in turn is following the protocol as described, to convince him that
the outcome is zero). Analogously, $P_{B}^{*}$ is the highest probability
of a malicious Bob convincing an honest Alice that he won. The bias
of a WCF protocol is defined as $\epsilon:=\max\left\{ P_{A}^{*},P_{B}^{*}\right\} -\frac{1}{2}$.
A protocol that is completely secure, has $\epsilon=0$ and one that
is completely insecure has $\epsilon=\frac{1}{2}$.

Using a classical channel of communication between Alice and Bob,
unless one makes further assumptions such as computational hardness
of certain problems or relativistic assumptions,\footnote{in terms of the spatial locations of the observers; not to be confused
with the term \emph{relativising} from computational complexity.} coin flipping (even weak) is impossible to implement with any security,
to wit: for all classical protocols at least one of the parties, viz.
a malicious Alice or a malicious Bob, can win with certainty because
one can show $\epsilon=\frac{1}{2}$ (viz. $\max\{P_{A}^{*},P_{B}^{*}\}=1$).
Using a quantum channel of communication, it was shown that WCF can
be implemented with vanishing bias. These works, however, do not account
for noise in their implementation. One path towards more robust security
is device independence wherein the players do not even trust their
devices (recall, they already do not trust the other party). This
is in contrast to the device independent setting considered in key
distribution where the two parties trust each other but neither their
devices nor the communication channel (TODO: is the classical communication
channel trusted?). }

\subsection{Contributions}

\branchcolor{purple}{{[}TODO: fix it---this is outdated{]} In this work, we start with
a device independent (DI) coin flipping (CF) protocol introduced\footnote{In fact, they introduced a device independent bit commitment protocol
which they in turn use to construct a strong coin flipping protocol
with the same cheating probabilities for Alice and Bob, $\approx0.854$
and $0.75$ respectively.} in \cite{Silman2011} which has $P_{A}^{*}=\cos^{2}(\pi/8)\approx0.854$
and $P_{B}^{*}=3/4=0.75$. They then compose these protocols to give
a balanced protocol, i.e. with $P_{A}^{*}=P_{B}^{*}\approx\frac{1}{2}+0.336$.
To the best of our knowledge, this DI CF protocol has the best security
guarantee. While Kitaev's bound for CF rules out perfect DI CF, no
lower bounds on the bias are known for DI WCF. In this work, however,
we focus on improving the upper bound on the bias, viz. we give DI
WCF protocols with biases $\approx0.319$.

We introduce two key new ideas which result in better protocols. The
first, is the use of self-testing by one party before initiating the
protocol and the second, is a more general technique to convert unbalanced
protocols (i.e. ones in which the probability of maliciously winning
for Alice and Bob are unequal) into balanced ones.}

\subsection{Proof Technique\label{subsec:Proof-Technique}}

\subsubsection*{Notation and Cheat Vectors}

\branchcolor{purple}{We introduce some notation to facilitate the discussion here. Denote
the DI CF protocol introduced in \cite{Silman2011} by $\mathcal{I}$
and let $p_{A}^{*}(\mathcal{I})\approx0.853\dots$ denote the maximum
probability with which a malicious Alice can win against honest Bob
who is following the protocol $\mathcal{I}$ and similarly, let $p_{B}^{*}(\mathcal{I})\approx0.75$
denote the maximum probability with which a malicious Bob can win
against an honest Alice who is following the protocol $\mathcal{I}$. 

One of the key observations we make in this work is the use of what
we call ``cheat vectors''---it is any tuple of probabilities which
can arise in a CF protocol when one player is honest. More precisely,
suppose Alice is (possibly) malicious and Bob follows the protocol
$\mathcal{I}$. Then, the cheat vectors for Alice constitute the set
\begin{equation}
\mathbb{C}{}_{A}(\mathcal{I}):=\{(\alpha,\beta,\gamma):\exists\text{ a strategy for }A\text{ s.t. an honest }B\text{ outputs }\text{0,1, \text{ and }\ensuremath{\perp} with probabilities \ensuremath{\alpha,\beta} and \ensuremath{\gamma}}\}.\label{eq:cheatVectors}
\end{equation}
We analogously define $\mathbb{C}_{B}(\mathcal{I})$. Cheat vectors
become useful when we try to compose protocols. The observation then,
is that the abort event can be taken to abort the full protocol instead
of being treated as the honest player winning. The latter gives the
malicious player further opportunity to cheat and so preventing it
improves the security. }

\subsubsection*{Protocols}

\branchcolor{purple}{We introduce two variants of protocol $\mathcal{I}$, which we call
$\mathcal{P}$ and $\mathcal{Q}$. 
\begin{itemize}
\item $\mathcal{P}$ is essentially the same as $\mathcal{I}$ except that
Alice self-tests her boxes before starting the protocol and performs
an additional test to ensure Bob doesn't cheat. We show that $p_{A}^{*}(\mathcal{P})\apprle0.853\dots$
and $p_{B}^{*}(\mathcal{P})\apprle0.667\dots$. We also show that
$\mathbb{C}_{B}(\mathcal{P})$ can be cast as an SDP.
\item $\mathcal{Q}$ is also essentially the same as $\mathcal{I}$ except
that Bob self-tests his boxes before starting the protocol. In this
case, $p_{X}^{*}(\mathcal{Q})=p_{X}^{*}(\mathcal{I})$ for both values
of $X\in\{A,B\}$ so the advantage isn't manifest. However, now $\mathbb{C}_{A}(\mathcal{Q})$
can be cast as an SDP which, as we shall see, yields an advantage
when $\mathcal{Q}$ is composed. 
\end{itemize}
}

\subsubsection*{Compositions}

\branchcolor{purple}{As the protocols $\mathcal{X}\in\{\mathcal{I},\mathcal{P},\mathcal{Q}\}$
all have skewed security---either $p_{A}^{*}(\mathcal{X})>p_{B}^{*}(\mathcal{X})$
or the other way---and therefore the bias is determined by $p_{\max}^{*}(\mathcal{X}):=\max\{p_{A}^{*}(\mathcal{X}),p_{B}^{*}(\mathcal{X})\}$.
Note that, $p_{\max}^{*}(\mathcal{X})=p_{\max}^{*}(\mathcal{Y})$
for all $\mathcal{X},\mathcal{Y}\in\{\mathcal{I},\mathcal{P},\mathcal{Q}\}$,
which means that we don't immediately obtain an advantage. However,
the most obvious method of composing these protocols to obtain a new
protocol, which we describe later, ``balances'' the advantage. After
this composition procedure is applied to some protocol $\mathcal{X}$,
we denote the resulting protocol by $C_{LL}(\mathcal{X})$. Applying
this technique to $\mathcal{P}$, we already obtain a more secure
protocol.
\begin{itemize}
\item For all $X\in\{A,B\}$ the cheating probabilities for protocol $\mathcal{I}$
under the standard composition is given by
\[
p_{X}^{*}(C^{LL}(\mathcal{I}))\approx\frac{1}{2}+0.336\dots
\]
while for the improved protocol $\mathcal{P}$, these are given by
\begin{equation}
p_{X}^{*}(C^{LL}(\mathcal{P}))\approx\frac{1}{2}+0.3199\dots.\label{eq:SikoraP}
\end{equation}
\end{itemize}
The standard composition technique doesn't yield any improvement for
$\mathcal{Q}$ because the cheating probabilities are identical to
those of $\mathcal{I}$. We can extract an advantage by using a composition
technique that uses ``cheat vectors'' and the abort event. We describe
it in detail later but for now, we simply denote the new protocol
obtained using this improved ``abort phobic'' composition (of protocol
$\mathcal{X}$) by $C_{\perp L}(\mathcal{X})$ or $C_{L\perp}(\mathcal{X})$. 
\begin{itemize}
\item Applying the technique to $\mathcal{P}$, the cheating probabilities
become 
\[
p_{X}^{*}(C^{\perp L}(\mathcal{P}))\approx\frac{1}{2}+0.3148\dots
\]
which is a further improvement.
\item Using this technique on $\mathcal{Q}$, the cheating probabilities
become 
\[
p_{X}^{*}(C^{L\perp}(\mathcal{Q}))\approx\frac{1}{2}+0.3226\dots
\]
for all $X\in\{A,B\}$, which is worse than even \Eqref{SikoraP}. 
\item However, when we combine both these protocols to obtain (again, for
all $X\in\{A,B\}$)
\[
p_{X}^{*}(C^{L\perp}(\mathcal{Q},\mathcal{Q},\dots,\mathcal{Q},\mathcal{P}))\approx\frac{1}{2}+0.29104\dots
\]
where we use the same composition technique except that at the last
``level'' we use\footnote{$C^{\perp L}(\mathcal{P},\mathcal{P},\dots,\mathcal{P},\mathcal{Q})$
is strictly worse than considering $C^{\perp L}(\mathcal{P},\mathcal{P},\dots,\mathcal{P},\mathcal{P})$;
this should become evident later.} $\mathcal{P}$ instead of $\mathcal{Q}$. 
\end{itemize}
\begin{table}
\begin{centering}
\begin{tabular}{cc}
\toprule 
Protocol & Bias\tabularnewline
\midrule
\midrule 
$C^{LL}(\mathcal{W},\dots,\mathcal{W})$ & $0.336\dots$\tabularnewline
\midrule 
$C^{LL}(\mathcal{P},\dots,\mathcal{P})$ & $0.3199\dots$\tabularnewline
\midrule 
$C^{\perp L}(\mathcal{P},\dots,\mathcal{P})$ & $0.3148\dots$ \tabularnewline
\midrule 
$C^{L\perp}(\mathcal{Q},\dots,\mathcal{Q})$ & $0.3226\dots$\tabularnewline
\midrule 
$C^{L\perp}(\mathcal{Q},\dots,\mathcal{Q},\mathcal{P})$ & $0.29104\dots$\tabularnewline
\bottomrule
\end{tabular}
\par\end{centering}
\caption{}

\end{table}

}


\section{Device Independent Weak Coin Flipping protocols | State Of The Art}

\branchcolor{purple}{In the following, we first discuss how one can describe DI WCF protocols
in terms of the players exchanging ``boxes''---devices which take
classical inputs and give classical outputs. Subsequently we recall
the GHZ test and finally we use these to delineate the DI-CF due to
\cite{Silman2011}.}

\subsection{Device Independence and the Box Paradigm}

\branchcolor{purple}{We describe device independent protocols as classical protocols with
the one modification: we assume that the two parties can exchange
boxes and that the parties can shield their boxes (from the other
boxes i.e. the boxes can't communicate with each other once shielded).\footnote{TODO: Verify if this notion is in fact correct; I hope I'm not making
a major mistake somehow. I should be able to take the POVMs as tensor
products right, because I can change them at will, independent of
the others (and ensuring that there's no communication between them;
could they be somehow entangled, i.e. could it be that somehow the
measurement operators are themselves quantum correlated?); I would
like to reach the conclusion starting from the locality assumption.}

}
\begin{defn}[Box]
 \label{def:box}A \emph{box} is a device that takes an input $x\in\mathcal{X}$
and yields an outputs $a\in\mathcal{A}$ where $\mathcal{X}$ and
$\mathcal{A}$ are finite sets. Typically, a set of $n$ boxes, taking
inputs $x_{1},x_{2},\dots x_{n}$ and yielding outputs $a_{1},a_{2}\dots a_{n}$
are \emph{characterised} by a joint conditional probability distribution,
denoted by 
\[
p(a_{1},a_{2}\dots a_{n}|x_{1},x_{2}\dots x_{n}).
\]
Further, if $p(a_{1},a_{2}\dots a_{n}|x_{1},x_{2}\dots x_{n})=\tr\left[M_{a_{1}|x_{1}}^{1}\otimes M_{a_{2}|x_{2}}^{2}\dots\otimes M_{a_{n}|x_{n}}^{n}\rho\right]$
then we call the set of boxes, \emph{quantum boxes}, where for a fixed
$x'$ $\{M_{a'|x'}^{i}\}_{a'\in\mathcal{A}_{i}}$ constitute a POVM
for the $i$th subsystem, $\rho$ is a density matrix and their dimensions
are mutually consistent.
\end{defn}

Henceforth, we restrict ourselves to quantum boxes. 
\begin{defn}[Protocol in the box formalism]
 \label{def:BoxProtocol}A generic two-party protocol in the box
formalism has the following form:
\begin{enumerate}
\item Inputs:
\begin{enumerate}
\item Alice is given boxes $\Box_{1}^{A},\Box_{2}^{A}\dots\Box_{p}^{A}$
and Bob is given boxes $\Box_{1}^{B},\Box_{2}^{B},\dots\Box_{q}^{B}$. 
\item Alice is given a random string $r^{A}$ and Bob is given a random
string $r^{B}$ (of arbitrary but finite length).
\end{enumerate}
\item Structure: At each round of the protocol, the following is allowed.
\begin{enumerate}
\item Alice and Bob can locally perform arbitrary but finite time computations
on a Turing Machine. 
\item They can exchange classical strings/messages and boxes.
\end{enumerate}
\end{enumerate}
\branchcolor{purple}{A protocol in the box formalism is readily expressed as a protocol
which uses a (trusted) classical channel (i.e. they trust their classical
devices to reliably send/receive messages), untrusted quantum devices
and an untrusted quantum channel (i.e. a channel that can carry quantum
states but may be controlled by the adversary).}
\end{defn}

\begin{assumption}[Setup of Device Independent Two-Party Protocols]
 Alice and Bob 
\begin{enumerate}
\item both have private sources of randomness,
\item can send and receive classical messages over a (trusted) classical
channel,
\item can prevent parts of their untrusted quantum devices from communicating
with each other, and
\item have access to an untrusted quantum channel.
\end{enumerate}
\end{assumption}

\branchcolor{purple}{We restrict ourselves to a ``measure and exchange'' class of protocols---protocols
where Alice and Bob start with some pre-prepared states and subsequently,
only perform classical computation and quantum measurements locally
in conjunction with exchanging classical and quantum messages. More
precisely, we consider the following (likely restricted) class of
device independent protocols.}
\begin{defn}[Measure and Exchange (Device Independent Two-Party) Protocols]
\label{def:MEprotocol} A \emph{measure and exchange (device independent
two-party) protocol} has the following form:
\begin{enumerate}
\item Inputs:
\begin{enumerate}
\item Alice is given quantum registers $A_{1},A_{2},\dots A_{p}$ together
with POVMs\footnote{For concreteness, take the case of binary measurements. By $\{M_{a|x}^{A_{1}}\}_{a}$,
for instance, we mean $\{M_{0|x}^{A_{1}},M_{1|x}^{A_{1}}\}$ is a
POVM for $x\in\{0,1\}$. } 
\[
\{M_{a|x}^{A_{1}}\}_{a},\{M_{a|x}^{A_{2}}\}_{a},\dots\{M_{a|x}^{A_{p}}\}_{a}
\]
which act on them and Bob is, analogously, given quantum registers
$B_{1},B_{2},\dots B_{q}$ together with POVMs 
\[
\{M_{b|y}^{B_{1}}\}_{b},\{M_{b|y}^{B_{2}}\}_{b},\dots,\{M_{b|y}^{B_{q}}\}_{b}.
\]
Alice shields $A_{1},A_{2},\dots A_{p}$ (and the POVMs) from each
other and from Bob's lab. Bob similarly shields $B_{1},B_{2}\dots B_{q}$
(and the POVMs) from each other and from Alice's lab.
\item Alice is given a random string $r^{A}$ and Bob is given a random
string $r^{B}$ (of arbitrary but finite length).
\end{enumerate}
\item Structure: At each round of the protocol, the following is allowed.
\begin{enumerate}
\item Alice and Bob can locally perform arbitrary but finite time computations
on a Turing Machine.
\item They can exchange classical strings/messages.
\item Alice (for instance) can 
\begin{enumerate}
\item send a register $A_{l}$ and the encoding of her POVMs $\{M_{i}^{A_{l}}\}_{i}$
to Bob, or
\item receive a register $B_{m}$ and the encoding of the POVMs $\{M_{i}^{B_{m}}\}_{i}$. 
\end{enumerate}
Analogously for Bob. 
\end{enumerate}
\end{enumerate}
\end{defn}

It is clear that a protocol in the box formalism (\Defref{BoxProtocol})
which uses only quantum boxes (\Defref{box}) can be implemented as
a measure and exchange protocol (\Defref{MEprotocol}).

\subsection{The GHZ Test}

\branchcolor{purple}{Before we define the current best DI CF protocol, we briefly remind
the reader of the GHZ test, upon which the aforementioned protocol
depends, and set up some conventions.}
\begin{defn}
\label{def:GHZ-box}Suppose we are given three boxes, $\Box^{A},\Box^{B}$
and $\Box^{C}$, which accept binary inputs $a,b,c\in\{0,1\}$ and
produces binary output $x,y,z\in\{0,1\}$ respectively. The boxes
pass the GHZ test if $a\oplus b\oplus c=xyz\oplus1$, given the inputs
satisfy $x\oplus y\oplus z=1$.
\end{defn}

\begin{claim}
\label{claim:Quantum-boxes-pass}Quantum boxes pass the GHZ test with
certainty (even if they cannot communicate), for the state $\left|\psi\right\rangle _{ABC}=\frac{\left|000\right\rangle _{ABC}+\left|111\right\rangle _{ABC}}{\sqrt{2}}$,
and measurement\footnote{we added the identity so that the eigenvalues associated become $0,1$
instead of $-1,1$.} $\frac{\sigma_{x}+\mathbb{I}}{2}$ for input $0$ and $\frac{\sigma_{y}+\mathbb{I}}{2}$
for input $1$ (in the notation introduced earlier, $M_{0|0}^{A}=\left|+\right\rangle \left\langle +\right|,M_{1|0}^{A}=\left|-\right\rangle \left\langle -\right|$
and so on, where $\left|\pm\right\rangle =\frac{\left|0\right\rangle \pm\left|1\right\rangle }{\sqrt{2}}$).\footnote{TODO: Think: Should I add the classical value? This would require
me to add what it means to have a classical box.}
\end{claim}

\branchcolor{purple}{The proof is easier to see in the case where the outcomes are $\pm1$;
it follows from the observations that $\sigma_{y}\otimes\sigma_{y}\otimes\sigma_{y}\left|\psi\right\rangle =-\left|\psi\right\rangle $,
$\sigma_{x}\otimes\sigma_{x}\otimes\sigma_{x}\left|\psi\right\rangle =\left|\psi\right\rangle $
and the anti-commutation of $\sigma_{x}$ and $\sigma_{y}$ matrices,
i.e. $\sigma_{x}\sigma_{y}+\sigma_{y}\sigma_{x}=0$. }

\subsection{The Protocol}

\branchcolor{purple}{The best DI CF protocol known is the one introduced in \cite{Silman2011}.
While this is a protocol for SCF, and so also works as a WCF protocol,
we do not know of any better protocol for the latter.}
\begin{lyxalgorithm}[SCF, original]
 \label{alg:SCForiginal}Alice has one box and Bob has two boxes
(in the security analysis, we let the cheating player distribute the
boxes). Each box takes one binary input and gives one binary output. 
\begin{enumerate}
\item Alice chooses $x\in_{R}\{0,1\}$ and inputs it into her box to obtain
$a$. She chooses $r\in_{R}\{0,1\}$ to compute $s=a\oplus x.r$ and
sends $s$ to Bob.
\item Bob chooses $g\in_{R}\{0,1\}$ (for ``guess'') and sends it to Alice.
\item Alice sends $x$ and $a$ to Bob. They both compute the output $x\oplus g$. 
\item Test round
\begin{enumerate}
\item Bob tests if $s=a$ or $s=a\oplus x$. If the test fails, he aborts.
Bob chooses $b,c\in_{R}\{0,1\}$ such that $a\oplus b\oplus c=1$
and then performs a GHZ using $a,b,c$ as the inputs and $x,y,z$
as the output from the three boxes. He aborts if this test fails.
\end{enumerate}
\end{enumerate}
\end{lyxalgorithm}

From \Claimref{Quantum-boxes-pass}, it is clear that when both players
follow \Algref{SCForiginal} using GHZ boxes (\Defref{GHZ-box}),
Bob never aborts and they win with equal probabilities. The security
of the protocol is summarised next.
\begin{lem}[Security of SCF]
 \cite{Silman2011} Let $\mathcal{I}$ denote the protocol corresponding
to \Algref{SCForiginal}. Then, the success probability of cheating
Bob, $p_{B}^{*}(\mathcal{I})\le\frac{3}{4}$ and that of cheating
Alice, $p_{A}^{*}(\mathcal{I})\le\cos^{2}(\pi/8)$. Further, both
bounds are saturated by a quantum strategy which uses a GHZ state
and the honest player measures along the $\sigma_{x}/\sigma_{y}$
basis corresponding to input $0/1$ into the box. Cheating Alice measures
along $\sigma_{\hat{n}}$ for $\hat{n}=\frac{1}{\sqrt{2}}(\hat{x}+\hat{y})$
while cheating Bob measures his first box along $\sigma_{x}$ and
second along $\sigma_{y}$. \label{lem:SCFstandard}
\end{lem}

\branchcolor{purple}{Note that both players can cheat maximally assuming they share a GHZ
state and the honest player measures along the associated basis. This
entails that even though the cheating player could potentially tamper
with the boxes before handing them to the honest player, surprisingly,
exploiting this freedom does not offer any advantage to the cheating
player. }

\section{First Technique: Self-testing (single shot, unbalanced)}

\branchcolor{purple}{TODO: Assumption: No honest abort.

We make two observations. 

First, in \Algref{SCForiginal} only Bob performs the test round.
In WCF, there is a notion of Alice winning and Bob winning. Thus,
if $x\oplus g=0$, i.e. the outcome corresponding to ``Alice wins'',
we can imagine that Bob continues to perform the test to ensure (at
least to some extend) that Alice did not cheat. However, if $x\oplus g=1$,
i.e. the outcome corresponding to ``Bob wins'', we can require Alice
to now complete the GHZ test to ensure that Bob did not cheat. It
turns out that this does not lower $p_{B}^{*}$. Interestingly, the
best cheating strategy deviates from the GHZ state and measurements
for the honest player. We omit the details here (see TODO: write this
down somewhere) but mention this to motivate the following. 

Second, Alice (say) can harness the self-testing property of GHZ states
and measurements to ensure that Bob has not tampered with her boxes.
One way of proceeding is that $N$ copies of the supposedly correct
boxes are distributed. Alice now picks one out of these $N$ boxes
at random and asks Bob to send the associated two boxes to each $N-1$
box that Alice posses. Alice runs the GHZ test on each box and if
even one test fails, she declares that Bob cheated. This way, for
a large $N$, Alice can ensure with near certainty, that she has a
box containing the correct state and (which performs the correct)
measurements. Note that no such scheme can be concocted which simultaneously
self-tests Alice and Bob's boxes. More precisely, no such procedure
can ensure that Alice and Bob share a GHZ state (Alice one part, Bob
the other two, for instance) because this would mean perfect (or near
perfect) SCF is possible which is forbidden even in the device dependent
case. Kitaev showed that for any SCF protocol, $\epsilon\ge\frac{1}{\sqrt{2}}-\frac{1}{2}$.

Combining these two observations, results in an improvement in the
security for Alice. We obtain a protocol with $P_{A}^{*}\le3/4$,
which is the same as before, but $P_{B}^{*}\apprle0.667...$.}

\subsection{Cheat Vectors }

\branchcolor{purple}{As alluded to in \Subsecref{Proof-Technique}, using cheat vectors,
it is sometimes possible to compose protocols and obtain a lower bias
compared to protocols which are composed without using cheat vectors.
We describe such procedures in the next section, \Secref{Second-Technique}.
Here, we simply define cheat vectors and show that self-testing allows
one to express relevant optimisation problems over cheat vectors as
semi definite programmes. }
\begin{defn}[Cheat Vectors]
 \label{def:CheatVectors}Given a protocol $\mathcal{I}$, denote
by $\mathbb{C}_{B}(\mathcal{I})$ the set of \emph{cheat vectors}
for Bob, which is defined as follows :
\[
\mathbb{C}_{B}(\mathcal{I}):=\{(\alpha,\beta,\gamma):\exists\text{ a strategy of \ensuremath{B} s.t. an honest \ensuremath{A} outputs }\text{0,1, \text{ and }\ensuremath{\perp} with probabilities \ensuremath{\alpha,\beta} and \ensuremath{\gamma}}\}
\]
and analogously, denote by $\mathbb{C}_{A}(\mathcal{I})$ the set
of cheat vectors for Alice (see \Eqref{cheatVectors}).
\end{defn}


\subsection{Alice self-tests | Protocol $\mathcal{P}$}

\branchcolor{purple}{We begin with the case where Alice self-tests.}
In the honest implementation, the \emph{trio} of boxes used in the
following are characterised by the GHZ setup (see \Claimref{Quantum-boxes-pass}). 
\begin{lyxalgorithm}[Alice self-tests her boxes]
\label{alg:AliceSelfTestsHerBoxes}There are $N$ trios of boxes;
Alice has the first part and Bob has the remaining two parts, of each
trio. 
\begin{enumerate}
\item Alice selects a number $i\in_{R}\{1,2\dots N\}$ and sends it to Bob.
\item Bob sends his part of the trio of boxes corresponding to $\{1,2\dots N\}\backslash i$,
i.e. he sends all the boxes, except the ones corresponding to the
trio $i$.
\item Alice performs a GHZ test on all the trios labelled $\left\{ 1,2\dots N\right\} \backslash i$,
i.e. all the trios except the $i$th.
\end{enumerate}
We restrict ourselves to the $i$th trio. Alice has one box and Bob
has two boxes. Each box takes one binary input and gives one binary
output. 
\begin{enumerate}
\item Alice chooses $x\in_{R}\{0,1\}$ and inputs it into her box to obtain
$a$. She chooses $r\in_{R}\{0,1\}$ to compute $s=a\oplus x.r$ and
sends $s$ to Bob.
\item Bob chooses $g\in_{R}\{0,1\}$ (for ``guess'') and sends it to Alice.
\item Alice sends $x$ to Bob. They both compute the output $x\oplus g$. 
\item Test rounds:
\begin{enumerate}
\item If $x\oplus g=0$:\\
Alice sends $a$ to Bob.\\
Bob tests if $s=a$ or $s=a\oplus x$. If the test fails, he aborts.
Bob chooses $y,z\in_{R}\{0,1\}$ such that $x\oplus y\oplus z=1$
and then performs a GHZ using $x,y,z$ as the inputs and $a,b,c$
as the output from the three boxes. He aborts if this test fails.
\item Else, if $x\oplus g=1$:
\begin{enumerate}
\item Alice chooses $y,z\in_{R}\{0,1\}$ s.t. $x\oplus y\oplus z=1$ and
sends them to Bob.
\item Bob inputs $y,z$ into his boxes, obtains and sends $b,c$ to Alice.
\end{enumerate}
Alice tests if $x,y,z$ as inputs and $a,b,c$ as outputs, satisfy
the GHZ test. She aborts if this test fails.

\end{enumerate}
\end{enumerate}
\end{lyxalgorithm}

\begin{lem}
\label{lem:AliceSelfTests}Let $\mathcal{P}$ denote the protocol
corresponding to \Algref{AliceSelfTestsHerBoxes}. Then Alice's cheating
probability $p_{A}^{*}(\mathcal{P})\le\cos^{2}(\pi/8)\approx0.852$.
Further, let $c_{0},c_{1},c_{\perp}\in\mathbb{R}$, and $\mathbb{C}_{B}(\mathcal{P})$
be the set of cheat vectors for Bob. Then, as $N\to\infty$, the solution
to the optimisation problem $\max(c_{0}\alpha+c_{1}\beta+c_{\perp}\gamma)$
over $\mathbb{C}_{B}(\mathcal{Q})$ approaches that of a semi definite
programme. In particular, i.e. for $c_{0}=c_{\perp}=0$ and $c_{1}=1$,
$p_{B}^{*}(\mathcal{P})\apprle0.667...$ (in the limit). 
\end{lem}

We defer the proof to \Subsubsecref{SDP-when-Alice}. \branchcolor{purple}{The value for $p_{B}^{*}(\mathcal{P})$ was obtained by numerically
solving the corresponding semi definite programme while the analysis
for cheating Alice is the same as that of the original protocol. } 

\subsection{Bob self-tests | Protocol $\mathcal{Q}$}

\branchcolor{purple}{What if we modified the protocol and had Bob self-test his boxes?
Does that yield a better protocol? We address the first question now
and the second in the subsequent section.}
\begin{lyxalgorithm}[Bob self-tests his boxes]
\label{alg:BobSelfTests}Proceed exactly as in \Algref{AliceSelfTestsHerBoxes},
except for the self-testing where the rolls of Alice and Bob are reversed.
More explicitly, suppose there are $N$ trios of boxes; Alice has
the first part and Bob has the remaining two parts, of each trio. 
\begin{enumerate}
\item Bob selects a number $i\in_{R}\{1,2\dots N\}$ and sends it to Alice.
\item Alice sends her part of the trio of boxes corresponding to $\{1,2\dots N\}\backslash i$,
i.e. she sends all the boxes, except the ones corresponding to the
trio $i$.
\item Bob performs a GHZ test on all the trios labelled $\left\{ 1,2\dots N\right\} \backslash i$,
i.e. all the trios except the $i$th.
\end{enumerate}
Henceforth, proceed as in \Algref{AliceSelfTestsHerBoxes} after the
self-testing step.
\end{lyxalgorithm}

\branchcolor{purple}{As already indicated in \Subsecref{Proof-Technique}, we don't expect
the cheating probabilities to improve but we do expect an SDP characterisation
of Alice's cheat vectors.}
\begin{lem}
\label{lem:Bob-self-tests}Let $\mathcal{Q}$ denote the protocol
corresponding to \Algref{BobSelfTests}. Then, Alice's cheating probability,
$p_{A}^{*}(\mathcal{Q})\le3/4$ and Bob's cheating probability, $p_{B}^{*}(\mathcal{Q})\le\cos^{2}(\pi/8)$
(which are the same as those in \Lemref{SCFstandard}). Further, let
$c_{0},c_{1},c_{\perp}\in\mathbb{R}$, and $\mathbb{C}_{A}(\mathcal{Q})$
be the set of cheat vectors for Alice. Then, as $N\to\infty$, the
solution to the optimisation problem $\max(c_{0}\alpha+c_{1}\beta+c_{\perp}\gamma)$
over $(\alpha,\beta,\gamma)\in\mathbb{C}_{A}(\mathcal{Q})$ approaches
that of a semi definite programme. 
\end{lem}

The proof is again deferred; see \Subsecref{SDP-when-Bob}.

\section{Second Technique: Bias Suppression \label{sec:Second-Technique}}

In this section, we use the convention that $\mathcal{I},\mathcal{P}$
and $\mathcal{Q}$ correspond to the protocols described in \Algref{SCForiginal},
\Algref{AliceSelfTestsHerBoxes} and \Algref{BobSelfTests}, respectively.
\branchcolor{purple}{Notice that $p_{A}^{*}(\mathcal{X})>p_{B}^{*}(\mathcal{X})$ where
$\mathcal{X}\in\{\mathcal{I},\mathcal{P},\mathcal{Q}\}$. We call
such protocols ``unbalanced''. In this section we start from unbalanced
WCF protocols and compose them to construct balanced WCF protocols.
To this end, we introduce some notation and the term ``polarity'',
to capture which among $A$ and $B$ is favoured. }
\begin{defn}[Unbalanced protocols, Polarity]
 \label{def:unbalanced-polarity}Given a WCF protocol $\mathcal{X}$,
we say that it is unbalanced if $p_{A}^{*}(\mathcal{X})\neq p_{B}^{*}(\mathcal{X})$.
We say that $\mathcal{X}$ has polarity $A$ if $p_{A}^{*}(\mathcal{X})>p_{B}^{*}(\mathcal{X})$
and polarity $B$ if $p_{A}^{*}(\mathcal{X})<p_{B}^{*}(\mathcal{X})$. 

Finally, let $X,Y\in\{A,B\}$ be distinct and suppose that $\mathcal{R}$
is an unbalanced protocol. Then, we define $\mathcal{\mathcal{R}}_{X}$
to be protocol $\mathcal{R}$ where Alice's and Bob's roles are possibly
interchanged so that $\mathcal{R}_{X}$ has polarity $X$, i.e. $p_{X}^{*}(\mathcal{\mathcal{R}}_{X})>p_{Y}^{*}(\mathcal{R}_{X})$.
We refer to $\mathcal{R}_{X}$ as $\mathcal{R}$ polarised towards
$X$.
\end{defn}

\branchcolor{purple}{

We now describe how these protocols can be composed such that the
``winner gets polarity''. }

\subsection{Composition}
\begin{defn}[$C(.,.)$ and $C(.)$]
 \label{def:C}Given two unbalanced WCF protocols, $\mathcal{X}$
and $\mathcal{Y}$, let $\mathcal{X}_{A},\mathcal{X}_{B}$ and $\mathcal{Y}_{A},\mathcal{Y}_{B}$
be their polarisations (see \Defref{unbalanced-polarity}).  Define
$C(\mathcal{X},\mathcal{Y})$ as follows:
\begin{enumerate}
\item Alice and Bob execute $\mathcal{X}_{A}$ and obtain outcome $X\in\{A,B,\perp\}$. 
\item If
\begin{enumerate}
\item $X=A$, execute $\mathcal{Y}_{A}$ and obtain outcome $Y\in\{A,B,\perp\}$,
else if
\item $X=B$, execute $\mathcal{Y}_{B}$ and obtain outcome $Y\in\{A,B,\perp\}$,
and finally if
\item $X=\perp$, set $Y=\perp$.
\end{enumerate}
Output $Y$.
\end{enumerate}
Let $\mathcal{Z}^{i+1}:=C(\mathcal{X},\mathcal{Z}^{i})$ for $i\ge1$,
and $\mathcal{Z}^{1}:=\mathcal{X}$. Then, formally, define $C(\mathcal{X}):=\lim_{i\to\infty}\mathcal{Z}^{i}$.\footnote{This is just to facilitate notation. This way the cheating probabilities
$p_{A}^{*}$ and $p_{B}^{*}$ converge and numerically this only takes
a few compositions to reach in our case.} 
\end{defn}

\branchcolor{purple}{The study of such composed protocols is simplified by assuming that
in an honest run, neither player outputs $\perp$ (abort), i.e. they
either output $A$ or $B$. We take a moment to explain this.

Consider any protocol $\mathcal{R}$ where, when both players are
honest, the probability of abort is zero. The protocols we have described
so far, satisfy this property, so long as we assume that honest players
can prepare perfect GHZ boxes. Such protocols are readily converted
into protocols where an honest player never outputs abort. 

For instance, suppose that in the execution of the aforementioned
protocol $\mathcal{R}$ (with no-honest-abort), Alice behaves honestly
but Bob is malicious. Suppose after interacting with Bob, Alice reaches
the conclusion that she must abort. Since she knows that if Bob was
honest, the outcome abort could not have arisen, she concludes that
Bob is cheating and declares herself the winner, i.e. she outputs
$A$. Similarly, when Bob is honest and after the interaction, reaches
the outcome abort, he knows Alice cheated and can therefore declare
himself the winner, i.e. output $B$. 

Whenever we modify a protocol so that an honest Alice (Bob) replaces
the outcome abort with Alice (Bob) winning, we say Alice (Bob) is
\emph{lenient}. This is motivated by the fact that when we compose
protocols, if Alice can conclude that Bob is cheating, and she still
outputs $A$ instead of aborting, she is giving Bob further opportunity
to cheat---she is being lenient.}
\begin{defn}[$\mathcal{R}$ with lenient players]
\label{def:lenientR} Suppose $\mathcal{R}$ is a WCF protocol such
that when both players are honest, the probability of outcome abort,
$\perp$, is zero. Then by ``\emph{$\mathcal{R}$ with lenient Alice
(Bob)}'', which we denote by $\mathcal{R}^{L\perp}$ ($\mathcal{R}^{\perp L}$),
we mean that Alice (Bob) follows $\mathcal{R}$ except that the outcome
$\perp$ replaced with $A$ ($B$). Finally, by ``\emph{lenient $\mathcal{R}$}'',
which we denote by $\mathcal{R}^{LL}$, we mean $\mathcal{R}$ with
lenient Alice and Bob. 
\end{defn}

\branchcolor{purple}{For clarity and conciseness, we define $C^{LL}$ to be compositions
with lenient variants of the given protocols. We work out some examples
of such protocols and analyse their security in the following section.
These can be improved by considering $C^{L\perp}$ and $C^{\perp L}$---compositions
where only one player is lenient. We discuss those afterwards.}
\begin{defn}[$C^{LL}$, $C^{\perp L}$ and $C^{L\perp}$]
 Suppose a WCF protocol $\mathcal{X}$ can be transformed into its
\emph{lenient} variants (see \Defref{lenientR}). Then define 
\begin{align*}
C^{LL}(\mathcal{X},\mathcal{Y}) & :=C(\mathcal{X}^{LL},\mathcal{Y}),\\
C^{\perp L}(\mathcal{X},\mathcal{Y}) & :=C(\mathcal{X}^{\perp L},\mathcal{Y}),\quad\text{and}\\
C^{L\perp}(\mathcal{X},\mathcal{Y}) & :=C(\mathcal{X}^{L\perp},\mathcal{Y}).
\end{align*}
In words, $C^{LL}$ is referred to as a \emph{standard }composition,
while $C^{\perp L}$ and $C^{L\perp}$ are referred to as \emph{abort-phobic}
compositions. 
\end{defn}


\subsection{Standard Composition | $C^{LL}$}

\branchcolor{purple}{We begin with the simplest case, standard composition, $C^{LL}$.
Let us take an example. Consider protocol $\mathcal{P}$ (see \Algref{AliceSelfTestsHerBoxes})
and recall (see \Lemref{AliceSelfTests})
\begin{align*}
p_{A}^{*}(\mathcal{P}_{A}) & =:\alpha\approx0.852\dots,\\
p_{B}^{*}(\mathcal{P}_{A}) & =:\beta\approx0.667\dots.
\end{align*}
Note that therefore $p_{A}^{*}(\mathcal{P}_{B})=\beta$ and $p_{B}^{*}(\mathcal{P}_{B})=\alpha$.
Further, let $\mathcal{P}':=C^{LL}(\mathcal{P},\mathcal{P})$, i.e.
Alice and Bob (who are both lenient) first execute $\mathcal{P}_{A}$
and if the outcome is $A$, they execute $\mathcal{P}_{A}$, while
if the outcome is $B$, they execute $\mathcal{P}_{B}$. This is illustrated
in \Figref{Standard-composition-technique} where note that the event
abort doesn't appear due to the leniency assumption. This allows us
to evaluate the cheating probabilities for the resulting protocol
as 
\begin{align}
p_{A}^{*}(\mathcal{P}') & =\alpha\alpha+(1-\alpha)\beta=:\alpha^{(1)},\quad\text{and}\label{eq:pStarAPprimeA}\\
p_{B}^{*}(\mathcal{P}') & =\beta\alpha+(1-\beta)\beta=:\beta^{(1)}.\nonumber 
\end{align}
To see this, consider \Eqref{pStarAPprimeA}. Alice knows that if
she wins the first round, her probability of winning is $\alpha>\beta$.
She knows that in the first round, she can force the outcome $A$
with probability $\alpha$. From leniency, she knows that Bob would
output $B$ with the remaining probability.\footnote{Without leniency, this probability could have been shared between
the outcomes $\perp$ (abort) and $B$. Consequently, only upper bounds
could be obtained on $p_{A}^{*}(\mathcal{P}')$ and $p_{B}^{*}(\mathcal{P}')$
using only $\alpha$ and $\beta$ as security guarantees for $\mathcal{P}_{A}$.
Upper bounds, however, would not be enough to determine the polarity
of $\mathcal{P}'$ and an stymie unambiguous repetition of the composition
procedure (at least as it is defined). One could nevertheless compose
by hoping that the upper bounds can be used to accurately represent
the polarity. This would still yield a protocol and the same calculation
would yield correct bounds but the composition itself might be sub-optimal.}

A side remark: one consequence of this simplified analysis is that\footnote{$\alpha^{(1)}-\beta^{(1)}=(\alpha-\beta)\alpha-(\alpha-\beta)\beta=(\alpha-\beta)^{2}>0$}
$\alpha^{(1)}>\beta^{(1)}$. Intuitively, it means that plority is
preserved by the composition procedure. Proceeding similarly, i.e.
defining $\mathcal{\mathcal{P}}'':=C^{LL}(\mathcal{P},\mathcal{P}')$,
and repeating $k+1$ times overall, one obtains\footnote{Again, note that $\alpha^{(k+1)}-\beta^{(k+1)}=(\alpha^{(k)}-\beta^{(k)})(\alpha-\beta)>0$.
} 
\begin{align*}
\alpha^{(k+1)} & =\alpha\alpha^{(k)}+(1-\alpha)\beta^{(k)}\\
\beta^{(k+1)} & =\beta\alpha^{(k)}+(1-\beta)\beta^{(k)}.
\end{align*}
In the limit of $k\to\infty$, one obtains 
\[
p_{A}^{*}(C^{LL}(\mathcal{P}))=p_{B}^{*}(C^{LL}(\mathcal{P}))=\lim_{k\to\infty}\alpha^{(k)}=\lim_{k\to\infty}\beta^{(k)}\approx0.8199\dots.
\]
Proceeding similarly, one obtains for $X\in\{A,B\}$ and $\mathcal{X}\in\{\mathcal{I},\mathcal{Q}\}$,
\[
p_{X}^{*}(C^{LL}(\mathcal{X}))\approx0.836\dots
\]
We thus have the following.

\begin{figure}
\begin{centering}
\includegraphics[width=6cm]{figures/temp_standardComp}
\par\end{centering}
\caption{Standard analysis (TODO: remove the abort)\label{fig:Standard-composition-technique}}
\end{figure}
}
\begin{thm}
Let $X\in\{A,B\}$ and $\mathcal{X}\in\{\mathcal{I},\mathcal{Q}\}$.
Then $p_{X}^{*}(C^{LL}(\mathcal{P}))\approx0.8199\dots$ and $p_{X}^{*}(C^{LL}(\mathcal{X}))\approx0.836\dots$. 
\end{thm}


\subsection{Abort Phobic Compositions | $C^{L\perp},C^{\perp L}$}

\branchcolor{purple}{We now look at the case of abort phobic compositions, $C^{L\perp}$
and $C^{\perp L}$. We work through essentially the same example as
above and see what changes in this setting. Consider protocol $\mathcal{P}$
(see ...) and recall that as before 
\begin{align*}
p_{A}^{*}(\mathcal{P}_{A}) & =:\alpha\approx0.852\dots,\\
p_{B}^{*}(\mathcal{P}_{A}) & =:\beta\approx0.667\dots.
\end{align*}
In addition, we know from \Lemref{AliceSelfTests} that cheat vectors
for Bob, $(\alpha,\beta,\gamma)\in\mathbb{C}_{B}(\mathcal{P}_{A})$
admit a nice characterisation courtesy of the self testing step. Let
$\mathcal{P}':=C^{\perp L}(\mathcal{P},\mathcal{P})$, i.e. Alice
and Bob execute $\mathcal{P}_{A}$ and if the outcome is $A$, they
execute $\mathcal{P}_{A}$ while if the outcome is $B$, they execute
$\mathcal{P}_{B}$. Bob is assumed to be lenient so an honest Bob
never outputs abort, $\perp$. However, an honest Alice can output
abort, $\perp$ so we keep that output in the illustration, \Lemref{AliceSelfTests}.
Our goal is to find $p_{A}^{*}(\mathcal{P}')$ and $p_{B}^{*}(\mathcal{P}')$.
The former is the same as before because Bob is lenient: 
\[
p_{A}^{*}(\mathcal{P}')=\alpha\cdot\alpha+(1-\alpha)\cdot\beta.
\]
Clearly, $p_{B}^{*}(\mathcal{P}')\le\beta\alpha+(1-\beta)\beta$ but
this bound may not be tight because $(1-\beta)$ is the combined probability
of Alice aborting and Alice outputting $A$. However, we can use cheat
vectors to obtain 
\[
p_{B}^{*}(\mathcal{P}')=\max_{(v_{A},v_{B},v_{\perp})\in\mathbb{C}_{B}}v_{B}\alpha+v_{A}\beta
\]
which is an SDP one can solve numerically. Unlike the previous case,
the polarity of the resulting protocol, $\mathcal{P}'$, might have
flipped (compared to the polarity of $\mathcal{P}$). 

Repeating this procedure, one can consider $\mathcal{P}'':=C^{\perp L}(\mathcal{P},\mathcal{P}')$
and obtain $p_{A}^{*}(\mathcal{P}'')$ directly as illustrated above
and numerically solve for $p_{B}^{*}(\mathcal{P}'')$ using the cheat
vectors. Numerically, we found that ten-fifteen repetitions caused
the cheating probabilities to converge to approximately $0.81459$.
We saw that the abort probabilities associated with $\mathcal{P}$
were quite small and therefore one could hope that $\mathcal{Q}$
fares better. Proceed analogously for protocol and considering $\mathcal{Q}':=C^{L\perp}(\mathcal{Q},\mathcal{Q})$,
$\mathcal{Q}'':=C^{L\perp}(\mathcal{Q},\mathcal{Q}')$, etc., the
cheating probabilities converge to approximately $0.822655$. 

\begin{figure}
\begin{centering}
\includegraphics[width=6cm]{figures/temp_AAC_Ps}
\par\end{centering}
\caption{Cheat vector analysis. (TODO: improve the caption)\protect \\
$(v_{A},v_{B},v_{\perp})\in\mathbb{C}_{B}$;  \label{fig:Abort-Augmented-Composition}}
\end{figure}
}
\begin{thm}
Let $X\in\{A,B\}$. Then 
\[
p_{X}^{*}(C^{\perp L}(\mathcal{P}))\approx0.81459
\]
and 
\[
p_{X}^{*}(C^{L\perp}(\mathcal{Q}))\approx0.822655
\]
where the latter holds assuming Conjecture ?? is true. 
\end{thm}

\branchcolor{purple}{While by itself $\mathcal{Q}$ doesn't seem to help, one can suppress
the bias further, by noting that at the very last step, only the cheating
probabilities $p_{A}^{*}(\mathcal{Q})$ and $p_{B}^{*}(\mathcal{Q})$
played a role (i.e. the fact that the cheating vectors $\mathbb{C}_{A}$
for $\mathcal{Q}$ had an SDP characterisation was not used). Further,
we know that $p_{A}^{*}(\mathcal{P})=p_{A}^{*}(\mathcal{Q})$ but
$p_{B}^{*}(\mathcal{P})<p_{B}^{*}(\mathcal{Q})$, i.e. using $\mathcal{P}$
at the very last step will result in a strictly better protocol. }
\begin{thm}
Let $X\in\{A,B\}$, 
\begin{align*}
\mathcal{Z}^{1} & :=C(\mathcal{Q},\mathcal{P}),\quad{\rm and}\\
\mathcal{Z}^{i+1} & :=C(\mathcal{Q},\mathcal{Z}^{i})\quad i>1.
\end{align*}
Then 
\[
\lim_{i\to\infty}p_{X}^{*}(\mathcal{Z}^{i})\approx0.791044\dots
\]
assuming Conjecture ?? holds. 
\end{thm}


\section{Security Proof | Asymptotic}

In this section, we prove the security under the following assumption:
\begin{assumption}
\label{assu:asymptotic}In protocol $\mathcal{P}$ ($\mathcal{Q}$),
Alice (Bob) does not perform the box verification step and instead
it is assumed that her box is (his boxes are) taken from a trio of
boxes which win the GHZ game with certainty. 
\end{assumption}

\branchcolor{purple}{Later, we drop the assumption and use the box verification step (see
..) to estimate the probability of winning the GHZ game. When the
winning probability is exactly one, the states and measurements are
the same as the GHZ state and $\sigma_{x},\sigma_{y}$ measurements,
up to local isometries and this allows us to use semi definite programming. }
\begin{lem}
\label{lem:self-testAsymptotic}Let $a,b,c,x,y,z\in\{0,1\}$. Consider
a trio of quantum boxes, specified by projectors $\{M_{a|x}^{A},M_{b|y}^{B},M_{c|z}^{C}\}$
acting on finite dimensional Hilbert spaces $\mathcal{H}^{A},\mathcal{H}^{B}$
and $\mathcal{H}^{C}$, and $\left|\psi\right\rangle \in\mathcal{H}^{A}\otimes\mathcal{H}^{B}\otimes\mathcal{H}^{C}=:\mathcal{H}^{ABC}$.
If the trio pass the GHZ test with certainty, then there exists a
local isometry 
\[
\Phi=\Phi^{A}\otimes\Phi^{B}\otimes\Phi^{C}:\mathcal{H}^{ABC}\to\mathcal{H}^{ABC}\otimes\mathbb{C}^{2\times3}
\]
such that 
\begin{align*}
\Phi\left(\left|\psi\right\rangle \right) & =\left|\chi\right\rangle \otimes\left|{\rm junk}\right\rangle ,\\
\Phi\left(M_{d|t}^{D}\left|\psi\right\rangle \right) & =\Pi_{d|t}^{D}\left|{\rm GHZ}\right\rangle \otimes\left|{\rm junk}\right\rangle \quad\forall D\in\{A,B,C\},\text{ and }d,t\in\{0,1\}
\end{align*}
where $\left|{\rm GHZ}\right\rangle =\frac{\left|000\right\rangle +\left|111\right\rangle }{\sqrt{2}}\in\mathbb{C}^{2\times3}$,
$\left|{\rm junk}\right\rangle \in\mathcal{H}^{ABC}$ is some arbitrary
state and $\{\Pi_{a|x}^{A},\Pi_{b|y}^{B},\Pi_{c|z}^{C}\}$ are projectors
corresponding to $\sigma_{x}$ on the first, second and third qubit
of $\left|{\rm GHZ}\right\rangle $ respectively, for $x=0$ and corresponding
to $\sigma_{y}$ for $x=1$, as in \Claimref{Quantum-boxes-pass}.
\end{lem}

\branchcolor{purple}{INTERNAL; (TODO: remove): Isometries can only increase dimensions
(they must be injective; that is to ensure they preserve inner products
of vectors). Therefore the isometry can't get rid of the $\left|{\rm junk}\right\rangle $
part. }

\subsection{Cheat vectors optimisation using Semi Definite Programming}

\subsubsection{SDP when Alice self-tests\label{subsubsec:SDP-when-Alice}}

\branchcolor{blue}{\begin{proof}[Asymptotic proof of \Lemref{AliceSelfTests}]
We prove \Lemref{AliceSelfTests} under \Assuref{asymptotic}. We
begin by making two observations. 

First, note that in the protocol, if Alice applies an isometry on
her box \emph{after} she has inputted $x$, obtained the outcome $a$
(and has noted it somewhere), the security of the resulting protocol
is unchanged because the rest of the protocol only depends on $x$
and $a$, and Alice's isometry only amounts to relabelling of the
post measurement state. This freedom allows us to simplify the analysis.

Second, in the analysis, we cannot model Alice's random choice, say
for $x$, as a mixed state because Bob can always hold a purification
and thus know $x$. Therefore, we model the randomness using pure
states and measure them in the end.

Notation: Other than $PQR$, all other registers store qubits.

We proceed step by step. 
\begin{enumerate}
\item We can model (justified below) Alice's act of inputting a random $x$
and obtaining an outcome $a$ from her box through the state 
\[
\left|\Psi_{0}\right\rangle :=\frac{1}{2}\sum_{x,a\in\{0,1\}}\left|xa\right\rangle _{XA}\left|\Phi(x,a)\right\rangle _{IJ}
\]
where $X$ represents the random input and $A$ the output. Here,
$\left|\Phi(x,a)\right\rangle _{IJ}$ are Bell states (see \Eqref{bellStates})
and the registers $IJ$ are held by Bob. Alice's act of choosing $r$
at random, computing $s=a\oplus x.r$ is modelled as 
\begin{equation}
\left|\Psi_{1}\right\rangle :=\frac{1}{2\sqrt{2}}\sum_{x,a,r\in\{0,1\}}\left|xa\right\rangle _{XA}\left|\Phi(x,a)\right\rangle _{IJ}\left|r\right\rangle _{R}\left|a\oplus x.r\right\rangle _{S}.\label{eq:Alice_Psi1}
\end{equation}
Finally, Alice's act of sending $s$ is modelled as Alice starting
with the state
\[
\tr_{IJS}\left[\left|\Psi_{1}\right\rangle \left\langle \Psi_{1}\right|\right]\in XAR.
\]

\branchcolor{blue}{\textbf{Justification for starting with $\left|\Psi_{0}\right\rangle $.}\\
To see why we start with the state $\left|\Psi_{0}\right\rangle $,
model Alice's choice of $x$ as $\left|+\right\rangle _{X}$, suppose
her measurement result is stored in $\left|0\right\rangle _{A}$,
the state of the boxes before measurement is $\left|\psi\right\rangle _{PQR}$
and Alice holds $P$, i.e. 
\[
\left|\Psi_{0}'\right\rangle :=\left|+\right\rangle _{X}\left|0\right\rangle _{A}\left|\psi\right\rangle _{PQR}.
\]
Let $\{M_{a|x}^{P}\}$ be the measurement operators corresponding
to Alice's box. The measurement process is unitarily modelled as 
\[
\left|\Psi_{1}'\right\rangle :=U_{{\rm measure}}\left|\Psi_{0}'\right\rangle =\frac{1}{\sqrt{2}}\sum_{x,a\in\{0,1\}}\left|x\right\rangle _{X}\left|a\right\rangle _{A}M_{a|x}^{P}\left|\psi\right\rangle _{PQR}
\]
where
\[
U_{{\rm measure}}=\sum_{x\in\{0,1\}}\left|x\right\rangle \left\langle x\right|_{X}\otimes\left[\mathbb{I}_{A}\otimes M_{0|x}^{P}+X_{X}\otimes M_{1|x}^{P}\right]\otimes\mathbb{I}_{QR}.
\]
Now we harness the freedom of applying an isometry to the post measured
state (as observed above). We choose the local isometry in \Lemref{self-testAsymptotic}.
Without loss of generality, we can assume that Bob had already applied
his part of the isometry before sending the boxes (because he can
always reverse it when it is his turn). We thus have, 
\begin{align*}
\left|\Psi_{2}'\right\rangle :=\Phi_{PQR}\left|\Psi_{1}'\right\rangle  & =\frac{1}{\sqrt{2}}\sum_{x,a\in\{0,1\}}\left|x\right\rangle _{X}\left|a\right\rangle _{A}\Pi_{x|a}^{H}\left|{\rm GHZ}\right\rangle _{HIJ}\otimes\left|{\rm junk}\right\rangle _{PQR}\\
 & =\frac{1}{2}\sum_{x,a\in\{0,1\}}\left|x\right\rangle _{X}\left|a\right\rangle _{A}U^{H}(x,a)\left|0\right\rangle _{H}\left|\Phi(x,a)\right\rangle _{IJ}\otimes\left|{\rm junk}\right\rangle _{PQR}
\end{align*}
where 
\begin{equation}
\left|\Phi(x,a)\right\rangle _{IJ}=\frac{\left|00\right\rangle +(-1)^{a}(i)^{x}\left|11\right\rangle }{\sqrt{2}}\label{eq:bellStates}
\end{equation}
 and $U^{H}(x,a)\left|0\right\rangle _{H}$ is $\frac{\left|0\right\rangle +(-1)^{a}(i)^{x}\left|1\right\rangle }{\sqrt{2}}$.
Since the state of register $H$ is completely determined by registers
$X$ and $A$, we can drop it from the analysis without loss of generality.
Finally, since $\left|{\rm junk}\right\rangle _{PQR}$ is completely
tensored out, we can drop it too without affecting the security. Formally,
we can assume that Alice gives Bob the register $P$ at this point.
}
\item Bob sending $g$ is modelled by introducing $\rho_{2}\in XARG$ satisfying
$\tr_{IJS}\left[\left|\Psi_{1}\right\rangle \left\langle \Psi_{1}\right|\right]=\tr_{G}(\rho_{2})$. 
\item At this point, either $x\oplus g$ is zero, in which case Alice's
output is fixed or $x\oplus g$ is one, and in that case Bob will
already know $x$ because he knows $g$ (he sent it) and Alice will
proceed to testing Bob. Formally, therefore, we needn't do anything
at this step.
\item Assuming $x\oplus g=1$, Alice sends $y,z$ to Bob such that $x\oplus y\oplus z=1$.
However, since Bob already knows $x$, he can deduce $z$ from $y$.
We thus only need to model Alice sending $y$ and Bob responding with
$d=b\oplus c$ (because Alice will only use $b\oplus c$ to test the
GHZ game, so it suffices for Bob to send $d$). This amounts to introducing
$\rho_{3}\in XARGYD$ satisfying $\rho_{2}\otimes\frac{\mathbb{I}_{Y}}{2}=\tr_{D}(\rho_{3})$. 
\item Since we postponed the measurements to the end, we add this last step.
Alice now measures $\rho_{3}$ to determine $x\oplus g$ and if it
is one, whether the GHZ test passed. Let 
\begin{align}
\Pi_{i} & :=\sum_{x,y\in\{0,1\}:x\oplus g=i}\left|xg\right\rangle \left\langle xg\right|_{XG}\otimes\mathbb{I}_{ARYD},\nonumber \\
\Pi^{{\rm GHZ}} & :=\sum_{\substack{x,y\in\{0,1\},\\
a,d\in\{0,1\}:a\oplus d\oplus1=xy\cdot(1\oplus x\oplus y)
}
}\left|xyad\right\rangle \left\langle xyad\right|_{XYAD}\otimes\mathbb{I}_{RG}.\label{eq:AliceProjs}
\end{align}
Then, we can write the cheat vector for Alice, i.e. the tuple of probabilities
that Alice outputs 0, 1 and abort (see \Defref{CheatVectors}), as
\[
(\alpha,\beta,\gamma)=(\tr(\Pi_{0}\rho_{3}),\tr(\Pi_{1}\Pi^{{\rm GHZ}}\rho_{3}),\tr(\Pi_{1}\bar{\Pi}^{{\rm GHZ}}\rho_{3}))
\]
 where $\bar{\Pi}:=\mathbb{I}-\Pi$.
\end{enumerate}
To summarise, the final SDP is as follows: let $\left|\Psi_{1}\right\rangle \in XAIJRS$
be as given in \Eqref{Alice_Psi1}, $\rho_{2}\in XARG$ and $\rho_{3}\in XARGYD$
\[
\max\quad\tr([c_{0}\Pi_{0}+\Pi_{1}(c_{1}\Pi^{{\rm GHZ}}+c_{\perp}\bar{\Pi}^{{\rm GHZ}})]\rho_{3})
\]
subject to 
\begin{align*}
\tr_{IJS}\left[\left|\Psi_{1}\right\rangle \left\langle \Psi_{1}\right|\right] & =\tr_{G}(\rho_{2})\\
\rho_{2}\otimes\frac{\mathbb{I}_{Y}}{2} & =\tr_{D}(\rho_{3})
\end{align*}
where the projectors are defined in \Eqref{AliceProjs}.
\end{proof}
}


\subsubsection{SDP when Bob self-tests\label{subsec:SDP-when-Bob}}

\branchcolor{blue}{\begin{proof}[Proof of \Algref{BobSelfTests}]
Denote by $\mathcal{I}$ the protocol corresponding to \Algref{SCForiginal}. 

It is evident that $p_{B}^{*}(\mathcal{Q})\le p_{B}^{*}(\mathcal{I})$
because compared to $\mathcal{I}$, in $\mathcal{Q}$ Alice performs
an extra test. However, it is not hard to see that the inequality
is saturated, i.e. $p_{B}^{*}(\mathcal{Q})=p_{B}^{*}(\mathcal{I})$.
Consider ... (TODO: recall/re-construct the cheating strategy for
Bob that lets him win with the same $3/4$ probability). 

From \Lemref{SCFstandard}, it is also clear that $p_{A}^{*}(\mathcal{Q})=p_{A}^{*}(\mathcal{I})$
because the only difference between Bob's actions in $\mathcal{Q}$
and $\mathcal{I}$ is that Bob self-tests to ensure his boxes are
indeed GHZ. However, the optimal cheating strategy for $\mathcal{I}$
can be implemented using GHZ boxes. 

This establishes the first part of the lemma. For the second part,
i.e. establishing that optimising $c_{0}\alpha+c_{1}\beta+c_{\perp}\gamma$
over $(\alpha,\beta,\gamma)\in\mathbb{C}_{A}$ is an SDP, we proceed
as follows. Suppose \Assuref{asymptotic} holds. Then we can assume
that Bob starts with the state 
\begin{equation}
\rho_{0}:=\tr_{H}(\left|{\rm GHZ}\right\rangle \left\langle {\rm GHZ}\right|_{HIJ})\label{eq:Bob_initState}
\end{equation}
 and the effect of measuring the two boxes can be represented by the
application of projectors of pauli operators $X$ and $Z$.

The justification is similar to that given in the former proof. Suppose
Bob holds registers $QR$ of $\left|\psi\right\rangle _{PQR}$ which
is the combined state of the three boxes. Suppose his measurement
operators are $\{M_{b|y}^{Q},M_{c|z}^{R}\}$. Then using the isometry
in \Lemref{self-testAsymptotic}, Bob can relabel his state (and without
loss of generality, we can suppose Alice also relabels according to
the aforementioned isometry) to get $\Phi_{PQR}\left|\psi\right\rangle _{PQR}=\left|{\rm GHZ}\right\rangle _{HIJ}\otimes\left|{\rm junk}\right\rangle _{PQR}$.
Further, since $\Phi_{PQR}(M_{b|y}^{Q}\otimes M_{c|z}^{R}\left|\psi\right\rangle _{PQR})=\Pi_{b|y}^{I}\Pi_{c|z}^{J}\left|{\rm GHZ}\right\rangle _{HIJ}\otimes\left|{\rm junk}\right\rangle _{PQR}$
Bob's act of measurement, in the new labelling, corresponds to simply
measuring the GHZ state in the appropriate Pauli basis. 
\begin{enumerate}
\item Bob receiving $s$ from Alice is modelled by introducing $\rho_{1}\in SIJ$
satisfying $\tr_{S}(\rho_{1})=\rho_{0}$. 
\item Bob sending $g\in_{R}\{0,1\}$ can be seen as appending a mixed state:
$\rho_{1}\otimes\frac{1}{2}\mathbb{I}_{G}$.
\item Alice sending $x$ (and $a$) can be modelled as introducing $\rho_{2}\in AXSIJG$
satisfying $\tr_{A}(\rho_{2})=\rho_{1}\otimes\frac{\mathbb{I}_{G}}{2}$.
\item To model the GHZ test, introduce a register $Y$ in the state $\frac{\left|0\right\rangle _{Y}+\left|1\right\rangle _{Y}}{\sqrt{2}}$.
Recall that to perform the GHZ test, we need $x\oplus y\oplus z=1$
i.e. $z=1\oplus y\oplus x$. Further introduce registers $B$ and
$C$ to hold the measurement results, define 
\begin{equation}
U:=\sum_{y,x\in\{0,1\}}\left|yx\right\rangle \left\langle yx\right|_{YX}\otimes(\mathbb{I}_{B}\otimes\Pi_{0|y}^{I}+X_{B}\otimes\Pi_{1|y}^{I})\otimes(\mathbb{I}_{C}\otimes\Pi_{0|(1\oplus y\oplus x)}^{J}+X_{C}\otimes\Pi_{1|(1\oplus y\oplus x)}^{J})\otimes\mathbb{I}_{ASG}.\label{eq:Bob-u}
\end{equation}
By construction, $\rho_{3}:=U\left(\left|+\right\rangle \left\langle +\right|_{Y}\otimes\left|00\right\rangle \left\langle 00\right|_{BC}\otimes\rho_{2}\right)U^{\dagger}\in YBCAXSIJG$
models the measurement process. (TODO: this equality would become
approximately true...but perhaps the noise can be absorbed in $\rho_{0}$
with some argument)
\item Since we postponed the measurements to the end, we add this step.
Define 
\[
\Pi_{i}:=\sum_{x,g\in\{0,1\}:x\oplus g=i}\left|xg\right\rangle \left\langle xg\right|_{XG}\otimes\mathbb{I}_{YABSIJ}
\]
to determine who won. Define 
\[
\Pi^{{\rm sTest}}:=\sum_{s,a,x\in\{0,1\}:s=a\lor s=a\oplus x}\left|sax\right\rangle \left\langle sax\right|_{SAX}\otimes\mathbb{I}_{GYBCIJ}
\]
 to model the first test, i.e. $s$ should either be $a$ or $a\oplus x$.
Define 
\[
\Pi^{{\rm GHZ}}:=\sum_{\substack{x,y\in\{0,1\},\\
a,b,c\in\{0,1\}:a\oplus b\oplus c\oplus1=xy\cdot(1\oplus x\oplus y)
}
}\left|xyabc\right\rangle \left\langle xyabc\right|_{XYABC}\otimes\mathbb{I}_{GSIJ}
\]
to model the GHZ test. Let 
\begin{equation}
\Pi^{{\rm Test}}:=\Pi^{{\rm GHZ}}\Pi^{{\rm sTest}},\quad\bar{\Pi}^{{\rm Test}}:=\mathbb{I}-\Pi^{{\rm Test}}.\label{eq:BobProjs}
\end{equation}
One can then write the cheat vector for Bob, i.e. the tuple of probabilities
that Bob outputs $0,1$ and abort (see \Defref{CheatVectors}), as
\[
(\alpha,\beta,\gamma)=(\tr(\Pi_{0}\Pi^{{\rm Test}}\rho_{3}),\tr(\Pi_{1}\rho_{3}),\tr(\Pi_{0}\bar{\Pi}^{{\rm Test}}\rho_{3})).
\]
\end{enumerate}
To summarise, the final SDP is as follows: let $\rho_{0}\in IJ$ be
as defined in \Eqref{Bob_initState}, $\rho_{1}\in SIJ$ and $\rho_{2}\in AXSIJG$.
Then, 
\[
\max\quad\tr\left([\Pi_{0}(c_{0}\Pi^{{\rm Test}}+c_{\perp}\bar{\Pi}^{{\rm Test}})+c_{1}\Pi_{1}]U\left(\left|+00\right\rangle \left\langle +00\right|_{YBC}\otimes\rho_{2}\right)U^{\dagger}\right)
\]
subject to 
\begin{align*}
\tr_{S}(\rho_{1}) & =\rho_{0}\\
\tr_{A}(\rho_{2}) & =\frac{1}{2}\rho_{1}\otimes\mathbb{I}_{G}
\end{align*}
where $U$ is as defined in \Eqref{Bob-u} and the projectors as in
\Eqref{BobProjs}.

\end{proof}
}

\section{Security Analysis | Finite n}

\subsection{Alice self tests}

\branchcolor{purple}{The basic idea here is to treat the state and measurements inside
the boxes as variables which are optimised over, subject to the constraint
that they are $\epsilon$-close to the ideal GHZ state and measurements.
This ceases to be an SDP so we relax the constraint that the post
measurement states must arise from measuring some fixed state and
let them be arbitrary states. The requirement that these are close
to the ideal GHZ state and post measured states is still enforced.
When $\epsilon=0$, we recover the asymptotic SDP (and that is no
longer a relaxation). Since we only change the constant $\epsilon$,
convergence of the objective value of these SDPs is easy to show {[}JAMIE{]}.}

aoeu

\branchcolor{blue}{\begin{proof}
We first write exactly what is going on physically, except that we
take the liberty of ``renaming'', i.e. applying global isometries.
We treat the state $\left|\psi\right\rangle _{PQR}$ and the measurements
$\{M_{a|x}^{P}\}$ as variables. 
\begin{enumerate}
\item We begin as before with $\left|\Psi'_{0}\right\rangle $, 
\[
\left|\Psi_{0}'\right\rangle :=\left|+\right\rangle _{X}\left|0\right\rangle _{A}\left|\psi\right\rangle _{PQR'}
\]
and\footnote{(we put $R'$ because we already used $R$ for the random register)}
obtain the ``post measurement state'' as 
\[
\left|\Psi_{1}'\right\rangle =\frac{1}{2}\sum_{x,a\in\{0,1\}}\left|x\right\rangle _{X}\left|a\right\rangle _{A}M_{a|x}^{P}\left|\psi\right\rangle _{PQR'}.
\]
Since we are allowed to ``rename'' (without changing the value of
the SDP), we have 
\begin{equation}
\left|\Psi_{2}'\right\rangle =\frac{1}{2}\sum_{x,a\in\{0,1\}}\left|x\right\rangle _{X}\left|a\right\rangle _{A}\Phi_{PQR}M_{a|x}^{P}\left|\psi\right\rangle _{PQR'}.\label{eq:PsiZeroBasically}
\end{equation}
At this point, in the asymptotic case, we could directly apply the
self-testing result and replace $\Phi_{PQR'}M_{a|x}^{P}\left|\psi\right\rangle _{PQR'}$
with $\Pi_{x|a}^{H}\left|{\rm GHZ}\right\rangle _{HIJ}\otimes\left|{\rm junk}\right\rangle _{PQR'}$.
Now, instead, we require 
\[
\left\Vert \Phi_{PQR'}M_{a|x}^{P}\left|\psi\right\rangle _{PQR'}-\Pi_{x|a}^{H}\left|{\rm GHZ}\right\rangle _{HIJ}\otimes\left|{\rm junk}\right\rangle _{PQR'}\right\Vert \le\epsilon\quad\forall a,x\in\{0,1\}
\]
{[}EDIT: here the norm is just the vector norm but we can impose it
as density matrices; there we use the trace norm; the remark shows
how to convert that into an SDP{]}where the norm\footnote{We could have used other norms but they would be a relaxation of the
constraints.} here is the trace norm $\left\Vert .\right\Vert $ (see \Remref{diffToNorm})
and $\epsilon'$ is a function $\epsilon$ which vanishes as $\epsilon$
vanishes ($\epsilon$ comes from the self testing step). One could,
henceforth, continue as in the asymptotic case. More precisely, one
could start with $\left|\Psi_{0}\right\rangle :=\left|\Psi_{2}'\right\rangle $,
model the classical computation step as 
\begin{align*}
\left|\Psi_{1}\right\rangle  & =U_{{\rm comp}}\left|\Psi_{0}\right\rangle \left|00\right\rangle _{RS}\\
 & =\frac{1}{2\sqrt{2}}\sum_{x,a,r\in\{0,1\}}\left|xa\right\rangle _{XA}\left|r\right\rangle _{R}\left|a\oplus x.r\right\rangle _{S}\Phi_{PQR}M_{a|x}^{P}\left|\psi\right\rangle _{PQR'}
\end{align*}
where $U_{{\rm comp}}$ is implicitly defined to yield the stated
state. Then, the act of sending $s$ (which is the first communication
step) is modelled as 
\[
\tr_{IJSPQR'}(\left|\Psi_{1}\right\rangle \left\langle \Psi_{1}\right|)\in XARH.
\]
\item The remaining steps are unchanged except that Alice additionally,
always holds the register $H$ now. 
\end{enumerate}
The final optimisation problem is defined on the variables $\left|\psi\right\rangle \in PQR'$,
$M_{a|x}^{P}$ projectors (or POVMs) acting on $PQR$, $\Phi_{PQR'}$
a local isometry\footnote{by local we mean it has the form $\Phi_{P}\otimes\Phi_{Q}\otimes\Phi_{R}$
where, for instance, $\Phi_{P}:P\to HP$.} from $PQR'\to HIJPQR'$, $\left|{\rm junk}\right\rangle \in PQR'$,
$\rho_{2}\in XARGH$ and $\rho_{3}\in XARGYDH$. The problem is:
\[
\max\quad\tr([c_{0}\Pi_{0}+\Pi_{1}(c_{1}\Pi^{{\rm GHZ}}+c_{\perp}\bar{\Pi}^{{\rm GHZ}})]\rho_{3})
\]
subject to 
\begin{align}
\left\Vert \Phi_{PQR'}M_{a|x}^{P}\left|\psi\right\rangle _{PQR'}-\Pi_{x|a}^{H}\left|{\rm GHZ}\right\rangle _{HIJ}\otimes\left|{\rm junk}\right\rangle _{PQR'}\right\Vert  & \le\epsilon\quad\forall a,x\in\{0,1\}\label{eq:NormNotSDP}\\
\left|\Psi_{1}\right\rangle  & :=U_{{\rm comp}}\left|\Psi_{0}\right\rangle \left|00\right\rangle _{RS}\nonumber \\
\tr_{IJSPQR'}\left[\left|\Psi_{1}\right\rangle \left\langle \Psi_{1}\right|\right] & =\tr_{G}(\rho_{2})\nonumber \\
\rho_{2}\otimes\frac{\mathbb{I}_{Y}}{2} & =\tr_{D}(\rho_{3})\nonumber 
\end{align}
where $\left|\Psi_{0}\right\rangle $ is as defined above (see \Eqref{PsiZeroBasically}
and recall that $\left|\Psi_{0}\right\rangle =\left|\Psi_{2}'\right\rangle $).
This, as it is stated, is not an SDP. However, it is clear that when
$\epsilon=0$, we recover the asymptotic case (many variables can
be dropped because they either are fixed (and no longer variable,
e.g. $\left|\Psi_{0}\right\rangle $) or become redundant, e.g. register
$H$). Let $v(\epsilon,d)$ be the value of the optimization program
above where $d$ encodes the dimension of systems $PQR$. We now relax
the constraints to obtain an SDP. Let $v'(\epsilon,d)$ be its value.
We want the relaxation to be such that $v'(0,d)=v(0,d)$. Additionally,
because it is a relaxation, we know $v(\epsilon,d)\le v'(\epsilon,d)$.
It then suffices to show the continuity of the relaxation (the SDP),
to establish the convergence of $v(\epsilon,d)$ to $v(0,d)$ as $\epsilon\to0$
{[}JAMIE double-check!{]}. 

There are two steps to the relaxation. First, we relax the \Eqref{NormNotSDP}
as in \Remref{TraceNormAsSDP}. This is straightforward. Second, we
remove the variables $\left|\psi\right\rangle ,M_{a|x}^{P}$ and $\Phi_{PQR'}$
and instead introduce variables $\left|\xi^{a,x}\right\rangle \in HIJPQR'$
for $a,x\in\{0,1\}$. We substitute\footnote{we can drop the pure state requirement; we use it for notational simplicity}
$\left|\psi\right\rangle $ with $\left|\xi\right\rangle $ and $M_{a|x}^{P}\left|\psi\right\rangle $
with $\left|\xi^{a,x}\right\rangle $ in the definition of $\left|\Psi_{0}\right\rangle $
and in the constraint \Eqref{NormNotSDP}. This is evidently a relaxation
(because one can represent any choice of $\left|\psi\right\rangle ,M_{a|x}^{P}$
and $\Phi_{PQR'}$ using $\left|\xi\right\rangle $ and $\left|\xi_{a,x}\right\rangle $
in the optimisation problem). Relaxing further to mixed states, the
SDP is then defined on $\xi^{aa',xx'}\in{\rm L}(HIJPQR')$ for $a,a',x,x'\in\{0,1\}$,
$\rho_{{\rm junk}}\in{\rm PSD}(PQR')$, $\rho_{2}\in{\rm PSD}(XARGH)$
and $\rho_{3}\in{\rm PSD}(XARGYDH)$ as 
\[
\max\quad\tr([c_{0}\Pi_{0}+\Pi_{1}(c_{1}\Pi^{{\rm GHZ}}+c_{\perp}\bar{\Pi}^{{\rm GHZ}})]\rho_{3})
\]
subject to 
\begin{align*}
\left\Vert \xi^{aa',xx'}-\Pi_{a|x}^{H}\left|{\rm GHZ}\right\rangle \left\langle {\rm GHZ}\right|_{HIJ}\Pi_{a'|x'}^{H}\otimes\rho_{{\rm junk}}\right\Vert  & \le\epsilon''\quad\forall a,a',x,x'\in\{0,1\}\\
\bar{\Psi}_{1} & :=U_{{\rm comp}}\bar{\Psi}_{0}\otimes\left|00\right\rangle \left\langle 00\right|_{RS}U_{{\rm comp}}^{\dagger}\\
\tr_{IJSPQR'}\left[\bar{\Psi}_{1}\right] & =\tr_{G}(\rho_{2})\\
\rho_{2}\otimes\frac{\mathbb{I}_{Y}}{2} & =\tr_{D}(\rho_{3})
\end{align*}
where 
\[
\bar{\Psi}_{0}:=\frac{1}{4}\sum_{x,x',a,a'\in\{0,1\}}\left|xa\right\rangle \left\langle x'a'\right|_{XA}\xi_{HIJPQR'}^{aa',xx'}
\]
and $\epsilon''$ is a function of $\epsilon$ which vanishes as $\epsilon$
vanishes. Clearly, when $\epsilon=0$, we recover the asymptotic SDP
and by construction, the SDP is a relaxation of the optimisation problem
we started with. Recall that $v'(\epsilon,d)$ is the value of this
SDP. It is easy {[}for JAMIE! please help{]} that $v'(\epsilon,d)$
is continuous as a function of $\epsilon$ (at least for small $\epsilon$?);
my guess would be that we are slowly enlarging the feasible region
so won't expect any jumps.
\end{proof}
}

aoeu
\begin{rem}
\label{rem:diffToNorm}It is straightforward to show that $\left\Vert \left|\rho\right\rangle -\left|\sigma\right\rangle \right\Vert \le\epsilon\implies\tr\left|\rho-\sigma\right|\le\epsilon'$
where $\epsilon'=2\sqrt{1-(1-\epsilon)}$. 

\label{rem:TraceNormAsSDP}For many norms (including the trace norm),
we have $\left\Vert X\right\Vert \le\epsilon\implies\left|\lambda_{\max}(X)\right|\le\epsilon'$
where $\epsilon'$ vanishes as $\epsilon$ vanishes. It is easy to
bound $\lambda_{\max}(X)\le\epsilon$ as 
\[
\left(\begin{array}{cc}
X-\epsilon\mathbb{I} & 0\\
0 & \epsilon\mathbb{I}-X
\end{array}\right)\ge0.
\]
This is an SDP constraint because we can define some $\left(\begin{array}{cc}
Y_{11} & Y_{12}\\
Y_{21} & Y_{22}
\end{array}\right)\ge0$ and then set the linear constraint $Y_{11}=X-\epsilon\mathbb{I}$
and $Y_{22}=\epsilon\mathbb{I}-X$. 

{[}EDIT When $X$ is not Hermitian, we can relax it using Schur's
complement as 
\[
\left(\begin{array}{cc}
\mathbb{I} & X\\
X^{T} & \epsilon''\mathbb{I}
\end{array}\right)\ge0\iff\epsilon''\mathbb{I}\ge X^{T}X
\]
and if $\left\Vert X\right\Vert \le\epsilon$, then there should be
some function $\epsilon''$ of $\epsilon$ that satisfies the above
(with possibly a multiplicative factor of $\dim(X)$). {]}
\end{rem}


\subsection{Bob self tests}
\begin{conjecture}[Continuity for $\mathcal{Q}$]
Blabla.
\end{conjecture}

For Bob's case, we work out an example which is essentially the same
as what we want to prove. In this case, we are unable to find a simple
SDP relaxation as above and instead rely on the NPA hierarchy for
the continuity result.
\begin{example}
We consider three optimisation problems. The first is supposed to
be the ``asymptotic version'', the second is supposed to by a toy
model of what happens in the lab with $\epsilon$ as a parameter,
and finally the third is an SDP relaxation of the second, obtained
using the NPA hierarchy. 

First: Let $\rho_{0}:=\tr_{HI}\left[\left|{\rm GHZ}\right\rangle \left\langle {\rm GHZ}\right|\right]_{HIJ}$.
The variable is $\rho_{1}\in ZJ$. The SDP program is 
\[
\max\quad\tr(\Pi_{{\rm obj}}\rho_{2}\Pi_{{\rm obj}})
\]
subject to 
\begin{align*}
\tr_{Z}(\rho_{1}) & =\rho_{0}\\
\rho_{2} & =\sum_{\substack{z,z'\\
c,c'
}
}\left|c\right\rangle \left\langle c'\right|_{C}\quad\otimes\quad\Pi_{c|z}^{J}\otimes\Pi_{z}^{Z}\quad\rho_{1}\quad\Pi_{c'|z'}^{J}\otimes\Pi_{z'}^{Z}
\end{align*}
where $\Pi_{{\rm obj}}$ is an arbitrary but fixed projector which
acts non-trivially on registers $CZ$ and $\{\Pi_{c|z}^{J}\}$ constitute
two sets of projective measurements, the setting indexed by $z$ and
outcome by $c$.

\branchcolor{purple}{The main simplifications we make, compared to Bob's asymptotic SDP,
are: \\
(1) we keep only the $J$ register from the $HIJ$ registers used
in the GHZ test, \\
(2) we skipped the part where Alice first sends $s$, then Bob sends
$g$ and in turn Alice sends $x$ and $a$ which are finally used
to do the test; we simply have her send $z$, the basis in which to
measure, \\
(3) the action of the (appropriately adapted) unitary $U$ is captured
directly by defining $\rho_{3}$\\
(4) the final measurement operator is left arbitrary so long as it
acts on ``classical registers'', $CZS$. 

These simplifications can be undone with the main idea unchanged.
We now proceed with defining the second variant which has the $PQR$
registers as well.}

Second: The variables are $\rho_{0}\in HIJPQR$, $\rho_{1}\in ZJR$,
$\rho_{{\rm junk}}\in R$ and $\{M_{0|z},M_{1|z}\}$ are projectors
acting on $JR$, for $z\in\{0,1\}$. The optimisation problem is 
\[
\max\quad\tr(\Pi_{{\rm obj}}\rho_{2}\Pi_{{\rm obj}})
\]
subject to 
\begin{align}
\left\Vert \rho_{0}-\left|{\rm GHZ}\right\rangle \left\langle {\rm GHZ}\right|_{HIJ}\otimes\rho_{{\rm junk}}\right\Vert  & \le\epsilon_{0}\label{eq:statePart}\\
\left\Vert M_{c|z}\rho_{0}M_{c'|z'}-\Pi_{c|z}^{J}\left|{\rm GHZ}\right\rangle \left\langle {\rm GHZ}\right|\Pi_{c'|z'}^{J}\otimes\rho_{{\rm junk}}\right\Vert  & \le\epsilon_{1}\quad\forall\quad c,c',z,z'\in\{0,1\}.\label{eq:MrhoM}\\
\tr_{Z}(\rho_{1}) & =\tr_{HIPQ}\rho_{0}\nonumber \\
\rho_{2} & =\sum_{\substack{z,z'\\
c,c'
}
}\left|c\right\rangle \left\langle c'\right|_{C}\quad\otimes\quad M_{c|z}\otimes\Pi_{z}^{Z}\quad\rho_{1}\quad M_{c'|z'}\otimes\Pi_{z'}^{Z}\nonumber 
\end{align}
where $\epsilon_{0}$ and $\epsilon_{1}$ are functions of $\epsilon$
which vanish as $\epsilon$ vanishes.

\branchcolor{purple}{We briefly justify why this optimization problem correctly captures
the physical situation, modulo the simplifications listed above (which
again, don't change the argument here). Let $\left|\psi\right\rangle \in HIJPQR$
be the state in the box and $M_{c|z}$ the measurement operators for
the last box. Since we're allowing Bob to optimise over $\left|\psi\right\rangle $
and $M_{c|z}$ we don't quite need to worry about the isometry in
the self-testing step. We suppress the $c$'s and $z$'s for the moment.
The self-testing statement says that $\left\Vert \left|\psi\right\rangle -\left|{\rm GHZ}\right\rangle \otimes\left|{\rm junk}\right\rangle \right\Vert \le\epsilon$
which entails \Eqref{statePart}. The self-testing statement also
says that $\left\Vert M\left|\psi\right\rangle -\Pi\left|{\rm GHZ}\right\rangle \otimes\left|{\rm junk}\right\rangle \right\Vert \le\epsilon$
which implies \Eqref{MrhoM}. 

It is straightforward to see that for $\epsilon=0$, this optimization
problem reduces to the first one. The $\rho_{0}$ part is trivial
and replacement of $M_{c|z}$ with $\Pi_{c|z}$ in $\rho_{2}$ can
be made as in illustrated \Exaref{replaceM} below.

}

Third: Denote the value of the second program by $v(\epsilon,d)$.
As argued, $v(0,d)$ is the value of the first program for all $d$
(finite $d$). {[}EDIT: I realised even an NPA relaxation is not simple/obvious
here{]} Let $w(\epsilon,d,k)$ denote the value of the NPA relaxation
of the second program, to level $k$. The NPA hierarchy is well known
and for our purposes here, it suffices to note two facts. First, the
NPA relaxation is always an SDP and second, the NPA relaxation converges
to, in this case, the second program as $k$ tends to infinity. Since
$v(\epsilon,d)\le w(\epsilon,d,k)$ for all $k$ and $d$, the continuity
result follows {[}JAMIE: complete the argument?{]}. 
\end{example}

\begin{example}
\label{exa:replaceM}Let $\rho_{AB}$ be a density matrix, $\Pi^{B},\Pi^{\prime B}$
be projectors on $B$ and $M^{B},M^{\prime B}$ be measurement (Kraus)
operators on $B$. Suppose $M^{B}\rho_{AB}M^{\prime B}=\Pi^{B}\rho_{AB}\Pi^{\prime B}$.
Suppose 
\begin{equation}
M^{B}\rho_{AB}M^{\prime B}=\Pi^{B}\rho_{Ab}\Pi^{\prime B}.\label{eq:measureMeasure}
\end{equation}
If $\sigma_{AB}$ is another density matrix such that $\tr_{A}(\sigma_{AB})=\tr_{B}(\rho_{AB})$,
then 
\begin{equation}
M^{B}\sigma_{AB}M^{\prime B}=\Pi^{B}\sigma_{AB}\Pi^{\prime B}.\label{eq:replaceMeasurement}
\end{equation}
 This follows from Uhlman's theorem which guarantees that there exists
a $U$ acting on system $A$ such that $\left(U\otimes\mathbb{I}_{B}\right)\sigma_{AB}\left(U^{\dagger}\otimes\mathbb{I}_{B}\right)=\rho_{AB}$.
Thus, conjugating \Eqref{measureMeasure} with $U\otimes\mathbb{I}_{B}$,
we obtain .
\end{example}


\subsection{Bob Self-tests | simplified}

\begin{cor}
\label{cor:UhlmanWithTrace}Let $\rho_{AB}\in D(AB)$ and $\rho'_{ABL}\in D(ABL)$
be two density matrices where $\dim L=\dim A\cdot\dim B$ {[}TODO:
check, I might need a constant{]}. Suppose $\rho'_{ABL}=\left|r'\right\rangle \left\langle r'\right|_{ABL}$
is pure and $\rho_{A}=\rho'_{A}$. Then there exists a unital map
$\mathcal{U}$ acting on registers $BL$ such that $\rho_{AB}=\mathcal{I}_{A}\otimes\tr_{L}\mathcal{U}_{BL}(\rho_{ABL}')$. 
\end{cor}

\begin{proof}
This follows from Uhlman's theorem. Let $\sigma_{ABL}=\left|s\right\rangle \left\langle s\right|_{ABL}$
be a purification of $\rho_{AB}$. Since $\sigma_{A}=\rho'_{A}$,
there exists a unitary $U_{AB}$ such that $\mathbb{I}\otimes U_{BL}\left|r'\right\rangle _{ABL}=\left|s\right\rangle _{ABL}$
(Uhlman's theorem). Tracing out $L$ yields the result. 
\end{proof}
Consider the following scenario:
\begin{itemize}
\item Bob has one box from a triple of GHZ boxes (assume he knows this box
is $\epsilon$ close to a GHZ box, up to some local isomorphism $\Phi$)
\item He receives a bit $z$ from Alice
\item He inputs the bit $z$ into the box, obtains an outcome \noun{$c$.}
\item Alice wants some function of $z,c$ maximized. Call this \emph{value}
$\eta_{\epsilon}^{{\rm lab}}$
\end{itemize}
Our objective is to show that as $\epsilon\to0$, $\eta_{\epsilon}^{{\rm lab}}=\eta$
where $\eta$ is the value of an SDP where Bob's box is replaced with
the GHZ state and measurement. We do this in three steps.
\begin{itemize}
\item We show $\eta_{\epsilon}^{{\rm lab}}\le\eta_{\epsilon}$ where $\eta_{\epsilon}$
does not depend explicitly depend on the isomorphism $\Phi$.
\item Then, we show that $\eta_{\epsilon}\le\eta_{\epsilon}^{{\rm Tr}}$
where $\eta_{\epsilon}^{{\rm Tr}}$ does not involve the ``junk''
space. 
\begin{itemize}
\item This also means $\eta_{\epsilon}^{{\rm Tr}}$ is continuous as a function
of $\epsilon$. 
\end{itemize}
\item And finally, we show $\eta_{0}^{{\rm Tr}}=\eta$ (where $\eta$ is
as defined above). 
\end{itemize}
In what follows, $PQR$ are arbitrary fixed dimensional spaces while
the remaining spaces denote qubits. 
\begin{defn}[$\eta_{\epsilon}^{{\rm lab}}$]
 Let $\rho_{0}\in D(PQR),\rho_{1}\in D(ZR),\rho_{{\rm junk}}\in D(PQR)$
be density matrices, $M_{c|z}\in{\rm Proj}(R)$ for $c,z\in\{0,1\}$
and $\Pi_{{\rm obj}}\in{\rm Proj}(CZ)$ be projectors\footnote{Take $\Pi_{{\rm obj}}$ to be a classical projector, i.e. linear combination
of $w_{cj}\Pi_{C}^{c}\otimes\Pi_{Z}^{z}$.} and finally, let $\Phi:PQR\to HIJPQR$ be a local isometry and let
$U\otimes V\otimes W$ denote its action, where $U^{\dagger}U=\mathbb{I}_{P}$
and so on. Then, define \label{def:etaLab}

\[
\eta_{\epsilon}^{{\rm lab}}:=\max\tr(\Pi_{{\rm obj}}\otimes\mathbb{I}_{R}\cdot\rho_{2})
\]
s.t. 
\begin{align*}
\left\Vert \Phi(\rho_{0})-\left|{\rm GHZ}\right\rangle \left\langle {\rm GHZ}\right|_{HIJ}\otimes\rho_{{\rm junk}}\right\Vert  & \le\epsilon\\
\left\Vert \Phi(M_{cz}\rho_{0}M_{c'z'})-\Pi_{cz}^{J}\left|{\rm GHZ}\right\rangle \left\langle {\rm GHZ}\right|_{HIJ}\Pi_{c'z'}^{J}\otimes\rho_{{\rm junk}}\right\Vert  & \le\epsilon & \forall c,c',z,z'\in\{0,1\}\\
\tr_{Z}(\rho_{1}) & =\tr_{PQ}(\rho_{0})\\
\rho_{2} & =\sum_{z,z',c,c'}\left|c\right\rangle \left\langle c'\right|_{C}\otimes\left(M_{cz}\otimes\Pi_{z}^{Z}\cdot\rho_{1}\cdot M_{c'z'}\otimes\Pi_{z'}^{Z}\right)
\end{align*}

\end{defn}

\begin{defn}[$\eta_{\epsilon}$]
 Let $\sigma_{0}\in D(HIJPQR),\sigma_{1}\in D(ZJR),\sigma_{{\rm junk}}\in D(PQR)$
be density matrices and $N_{c|z}\in{\rm Proj}(JR)$. Let $\Pi_{{\rm obj}}$,
$\Phi$, $U,V,W$ be as in \Defref{etaLab}. Then, define 
\[
\eta_{\epsilon}:=\max\tr(\Pi_{{\rm obj}}\otimes\mathbb{I}_{JR}\cdot\sigma_{2})
\]
s.t.
\begin{align*}
\left\Vert \sigma_{0}-\left|{\rm GHZ}\right\rangle \left\langle {\rm GHZ}\right|_{HIJ}\otimes\sigma_{{\rm junk}}\right\Vert  & \le\epsilon\\
\left\Vert N_{c|z}\sigma_{0}N_{c|z}-\Pi_{cz}^{J}\left|{\rm GHZ}\right\rangle \left\langle {\rm GHZ}\right|_{HIJ}\Pi_{cz}^{J}\otimes\sigma_{{\rm junk}}\right\Vert  & \le\epsilon & \forall c,c',z,z'\in\{0,1\}\\
\tr_{Z}(\sigma_{1}) & =\tr_{PQHI}(\sigma_{0})\\
\sigma_{2} & =\sum_{z,c}\left|c\right\rangle \left\langle c\right|_{C}\otimes\left(N_{cz}\otimes\Pi_{z}^{Z}\cdot\sigma_{1}\cdot N_{cz}\otimes\Pi_{z}^{Z}\right).
\end{align*}
\end{defn}

\begin{lem}
One has $\eta_{\epsilon}^{{\rm lab}}\le\eta_{\epsilon}$.
\end{lem}

\begin{proof}
{[}TODO: add details{]} Suppose $\rho_{0},\rho_{1},\rho_{2}$ and
$M_{cz}$ achieve the maximum for $\eta_{\epsilon}^{{\rm lab}}$.
Then $\sigma_{0}=V\rho_{0}V^{\dagger}$, $\sigma_{1}=V\rho_{1}V^{\dagger}$,
$\sigma_{2}=V\rho_{2}V^{\dagger}$ and $N_{cz}=VM_{cz}V^{\dagger}$,
observe that the constraints of $\eta_{\epsilon}$ are satisfied.
Further, observe that $\tr(\Pi_{{\rm obj}}\otimes\mathbb{I}_{IR}\sigma_{2})=\tr(\Pi_{{\rm obj}}\otimes\mathbb{I}_{R}\rho_{2})$;
the terms omitted from $\sigma_{2}$ do not contribute anyway because
of the structure of $\Pi_{{\rm obj}}$, i.e. it is a combination of
terms of the form $\Pi_{z}^{Z}\otimes\Pi_{c}^{C}$. Hence, $\eta_{\epsilon}$
is at least as large as $\eta_{\epsilon}^{{\rm lab}}$. 
\end{proof}
\begin{defn}[$\eta_{\epsilon}^{\rho}$]
 Let $c,z\in\{0,1\}$. Let $\tau_{HIJPQR}^{0,cz}\in D(HIJPQR)$,
$\tau_{ZJR}^{1,cz}\in D(ZJR)$, for all $c,z$, $\tau_{PQR}^{{\rm junk}}\in D(PQR)$
be density matrices and $\Pi_{{\rm obj}}\in{\rm Proj}(CZ)$ be as
in \Defref{etaLab}. Then define 
\[
\eta_{\epsilon}^{\rho}:=\max\quad\tr(\Pi_{{\rm obj}}\otimes\mathbb{I}_{JR}\cdot\tau_{CZJR}^{2})
\]
s.t. 
\begin{align*}
\left\Vert \tau_{HIJPQR}^{0,cz}-\Pi_{J}^{cz}\left|{\rm GHZ}\right\rangle \left\langle {\rm GHZ}\right|_{HIJ}\Pi_{J}^{cz}\otimes\tau_{PQR}^{{\rm junk}}\right\Vert  & \le\epsilon & \forall c,z\\
\tr_{Z}(\tau_{ZJR}^{1,cz}) & =\tr_{HIPQ}(\tau_{HIJPQR}^{0,cz}) & \forall c,z\\
\tau_{CZJR}^{2} & :=\sum_{z,c}\left|c\right\rangle \left\langle c\right|_{C}\otimes(\Pi_{Z}^{z}\cdot\tau_{ZJR}^{1,cz}\cdot\Pi_{Z}^{z}).
\end{align*}
\end{defn}

\begin{lem}
One has $\eta_{\epsilon}\le\eta_{\epsilon}^{\rho}$. 
\end{lem}

\begin{proof}
This is immediate. Suppose $\sigma_{0},\sigma_{1},\sigma_{2}$ and
$N^{cz}$ is a feasible point of $\eta_{\epsilon}$. Then, letting
$\tau^{0,cz}=N^{cz}\sigma_{0}N^{cz}$, $\tau^{1,cz}=N^{cz}\cdot\sigma_{1}\cdot N^{cz}$
it is clear that the constraints for $\eta_{\epsilon}^{\rho}$ are
satisfied. The objective function also have the same value by direct
substitution. 
\end{proof}
\begin{defn}[$\eta$]
 Let $\rho_{0}=\left|{\rm GHZ}\right\rangle \left\langle {\rm GHZ}\right|_{HIJ}$,
$\rho_{1}\in D(ZJ)$ be density matrices and $\Pi_{cz}^{J}\in{\rm Proj}(J)$
be GHZ scenario projectors (Pauli $x$ and Pauli $y$ projectors).
Let $\Pi_{{\rm obj}}$ be as in \Defref{etaLab}. Define 
\[
\eta:=\max\tr(\Pi_{{\rm obj}}\otimes\mathbb{I}_{J}\cdot\rho_{2})
\]
s.t. 
\begin{align*}
\tr_{Z}(\rho_{1}) & =\tr_{IJ}(\rho_{0})\\
\rho_{2} & =\sum_{z,z',c,c'}\left|c\right\rangle \left\langle c'\right|_{C}\otimes(\Pi_{cz}^{J}\otimes\Pi_{z}^{Z}\cdot\rho_{1}\cdot\Pi_{c'z'}^{J}\otimes\Pi_{z'}^{Z}).
\end{align*}
\end{defn}

\begin{lem}
One has $\eta_{\epsilon}^{\rho}=\eta$ when $\epsilon=0$. 
\end{lem}

\begin{proof}
Consider $\eta$. Note that one could, as before, replace $\rho_{2}$
with $\sum_{z,c}\left|c\right\rangle \left\langle c\right|_{C}\otimes(\Pi_{cz}^{J}\otimes\Pi_{z}^{Z}\cdot\rho_{1}\cdot\Pi_{cz}^{J}\otimes\Pi_{z}^{Z})$
without changing the value $\eta$ because $\Pi_{{\rm obj}}$ projects
on classical values of $C$ and $Z$. First, we show $\eta\le\eta_{\epsilon}^{\rho}$
when $\epsilon=0$. This is simple; suppose $\rho_{1}$ yields the
optimal value $\eta$. Let $\tau_{PQR}^{{\rm junk}}=1$, i.e. $PQR$
are zero dimensional. Let $\tau^{0,cz}=\Pi_{J}^{cz}\left|{\rm GHZ}\right\rangle \left\langle {\rm GHZ}\right|_{HIJ}\Pi_{J}^{cz}=\Pi_{J}^{cz}\rho_{0}\Pi_{J}^{cz}$.
Let $\tau^{1,cz}=\Pi_{J}^{cz}\rho_{1}\Pi_{J}^{cz}$. Evidently, $\tau^{2}$
and $\rho_{2}$ are identical and so the objective values are the
same. Now, we show that $\eta_{\epsilon}^{\rho}\le\eta$ when $\epsilon=0$.
Let $\{\tau^{1,cz}\}_{cz}$ be the optimal solution for $\eta_{\epsilon}^{\rho}$with
$\epsilon=0$. Note that this requires $PQR$ to be zero dimensional
and $\tau^{0,cz}=\Pi_{J}^{cz}\left|{\rm GHZ}\right\rangle \left\langle {\rm GHZ}\right|\Pi_{J}^{cz}$,
$\tr_{Z}(\tau^{1,cz})=\tr_{HI}(\Pi_{J}^{cz}\left|{\rm GHZ}\right\rangle \left\langle {\rm GHZ}\right|\Pi_{J}^{cz})$.
Let $\rho^{1}:=$
\end{proof}
{[}EDIT: The following is wrong; I am losing it; I somehow started
assuming that the measurement is somehow a unitary and arguing from
there.

EDIT2: I need to be more careful; do I get the asymptotic version
with $\epsilon=0$ if I trace out the extra part earlier and leave
something lingering in the density matrix? Suppose $\epsilon=0$ in
$\eta_{\epsilon}^{{\rm Tr}}$ below and say $\mathcal{P}$ is a projector.

{]}
\begin{defn}[$\eta_{\epsilon}^{{\rm Tr}}$]
 Let $\tau_{0}\in D(HIJ),\tau_{1}\in D(ZJL)$ be density matrices
and $\mathcal{P}_{JL}^{cz}$ be a unital map acting on $JL$. Here
$\dim L$ is only a function of the dimension of $\dim(HIJZ)$. Let
$\Xi_{J}^{cz}$ be the ``channel'' for $\Pi_{J}^{cz}$. Let $\Pi_{{\rm obj}}$
be as in \Defref{etaLab}. Then, define (indices are in superscripts;
not exponents)
\[
\eta_{\epsilon}^{{\rm Tr}}:=\max\tr(\Pi_{{\rm obj}}\otimes\mathbb{I}_{J}\cdot\tau_{CJZ}^{2})
\]
s.t.
\begin{align}
\left\Vert \tau_{HIJ}^{0}-\rho_{HIJ}^{{\rm GHZ}}\right\Vert  & \le\epsilon\nonumber \\
\left\Vert \mathcal{I}_{HI}\otimes\tr_{L}\mathcal{P}_{JL}^{c|z}(\tilde{\tau}_{HIJL}^{0})-\mathcal{I}_{HI}\otimes\Xi_{J}^{cz}(\rho_{HIJ}^{{\rm GHZ}})\right\Vert  & \le\epsilon & \forall c,z\in\{0,1\}\nonumber \\
\tr_{L}(\tilde{\tau}_{HIJL}^{0}) & =\tau_{HIJ}^{0}\\
\tr_{Z}(\tau_{JZ}^{1}) & =\tr_{HI}(\tau_{HIJ}^{0})\label{eq:constr1}\\
\tr_{L}(\tilde{\tau}_{JZL}^{1}) & =\tau_{JZ}^{1}\\
\tau_{JZC}^{2} & =\sum_{l,z,c}\left|c\right\rangle \left\langle c\right|_{C}\otimes\left(\Xi_{Z}^{z}\otimes\tr_{L}\mathcal{P}_{JL}^{cz}(\tilde{\tau}_{JZL}^{1})\right).\label{eq:constr2}
\end{align}
\end{defn}

\begin{lem}
One has $\eta_{\epsilon}\le\eta_{\epsilon}^{{\rm Tr}}$. 
\end{lem}

\branchcolor{blue}{
%
\begin{proof}
To simplify the proof, we first relax $\eta_{\epsilon}$ as (using
the channel notation; $N$ is denoted by $\mathcal{N}$, $\Pi$ by
$\Xi$, $\mathbb{I}$ by $\mathcal{I}$) 
\[
\eta_{\epsilon}':={\rm max}\quad\tr(\Xi_{{\rm obj}}\otimes\mathcal{I}_{J}(\xi_{JZC}))
\]
 
\begin{align*}
\left\Vert \xi_{HIJR}^{0}-\rho_{HIJ}^{{\rm GHZ}}\otimes\sigma_{R}^{{\rm junk}}\right\Vert  & \le\epsilon\\
\left\Vert \mathcal{I}_{HI}\otimes\mathcal{N}_{JR}^{c|z}(\xi^{0}{}_{HIJR})-\mathcal{I}_{HIR}\otimes\Xi_{J}^{cz}(\rho_{HIJ}^{{\rm GHZ}}\otimes\sigma_{R}^{{\rm junk}})\right\Vert  & \le\epsilon & \forall c,z\in\{0,1\}\\
\tr_{Z}(\xi_{JRZ}^{1}) & =\tr_{HI}(\xi_{HIJR}^{0})\\
\xi_{JZC}^{2} & =\sum_{c}\Pi_{C}^{c}\otimes\left(\Xi_{Z}^{z}\otimes\tr_{R}\mathcal{N}_{JR}^{c|z}(\xi_{JRZ}^{1})\right).
\end{align*}

Suppose $\sigma_{0},\sigma_{1},\sigma_{2}$ and $N_{c|z}$ are optimal
for $\eta_{\epsilon}$. Take $\xi_{HIJR}^{0}=\tr_{PQ}(\sigma_{0})$,
$\xi_{JRZ}^{1}=\tr_{PQ}(\sigma_{1})$ and $\xi_{JZC}^{2}=\tr_{PQR}(\sigma_{2})$.
Clearly, the objective function for $\eta_{\epsilon}'$ equals $\eta_{\epsilon}$
for the aforesaid choice of $\xi$s. The inequality constraints of
$\eta_{\epsilon}'$ follow from those of $\eta_{\epsilon}$ by monotonicity
of trace distance while the equality constraints follow from simply
tracing (and noting that $\mathcal{N}^{c|z}$ does not act on $PQ$).
Therefore $\xi$s are feasible. Thus $\eta_{\epsilon}'\ge\eta_{\epsilon}$. 

Suppose $\xi_{HIJR}^{0},\xi_{JRZ}^{1},\xi_{JZC}^{2}$ and $\mathcal{N}_{JR}^{c|z}$
yield an optimal solution to $\eta_{\epsilon}'$. Using these, we
first show to how construct $\mathcal{P}_{JL}^{cz}$. The remaining
argument is simple.
\begin{itemize}
\item Let $\bar{\xi}_{HIJRZ}^{1}$ be such that $\tr_{Z}\bar{\xi}_{HIJRZ}^{1}=\xi_{HIJR}^{0}$
and\footnote{(in the general case with more intermediate interaction but where
the boxes are measured in the end, take the ``purification'' of
the last state to include $HI$ which is untouched by the protocol
anyway)} $\tr_{HI}\bar{\xi}_{HIJRZ}^{1}=\xi_{JRZ}^{1}$. Let $\bar{\tau}_{HIJZL}^{1}$
be the purification of $\tr_{R}(\bar{\xi}_{HIJRZ}^{1})$. This is
where we restrict to a constant dimension.
\item There exists a $\mathcal{P}_{JL}$ which satisfies the following:
\begin{equation}
\mathcal{I}_{HIZ}\otimes\tr_{R}\mathcal{N}_{JR}^{cz}(\bar{\xi}_{HIJRZ}^{1})=\mathcal{I}_{HIZ}\otimes\tr_{L}\mathcal{P}_{JL}^{cz}(\bar{\tau}_{HIJZL}^{1}).\label{eq:purePure}
\end{equation}
This is guaranteed to exist because the LHS is a state in $D(HIJZ)$
which can be purified to a state in $D(HIJZL)$ and one can then apply
Uhlman's theorem (see \Corref{UhlmanWithTrace}). 
\end{itemize}
Let $\tau_{HIJ}^{0}=\tr_{R}\xi_{HIJR}^{0}$, $\tau_{JZ}^{1}=\tr_{R}\xi_{JRZ}^{1}$,
$\tilde{\tau}_{JZL}^{1}=\tr_{HI}(\bar{\tau}_{HIJZL}^{1})$ and $\tilde{\tau}_{HIJL}^{0}=\tr_{Z}(\bar{\tau}_{HIJZL}^{1})$.
We now show that $\tau^{0},\tau^{1},\tilde{\tau}^{1}$ and $\mathcal{P}_{JL}^{cz}$
are feasible for $\eta_{\epsilon}^{{\rm Tr}}$ and that the value
they yield is $\eta_{\epsilon}'$.

The first inequality constraint of $\eta_{\epsilon}^{{\rm Tr}}$ is
implied by that of $\eta_{\epsilon}'$ by monotonicity of trace distance
under quantum operations/channels; trace out $R$. Tracing out $R$
in the second inequality yields two terms, the first of which is $\mathcal{I}_{HI}\otimes\tr_{R}\mathcal{N}_{JR}^{cz}(\xi_{HIJR}^{0})=\mathcal{I}_{HI}\otimes{\rm tr}_{R}\mathcal{N}_{JR}^{cz}(\bar{\xi}_{HIJR}^{1})=\mathcal{I}_{HI}\otimes\tr_{L}\mathcal{P}_{JL}^{cz}(\bar{\tau}_{HIJL}^{1})=\mathcal{I}_{HI}\otimes\tr_{L}\mathcal{P}_{JL}^{cz}(\tilde{\tau}_{HIJL}^{0})$
using \Eqref{purePure} and the relations between $\xi^{0}$, $\bar{\xi}^{1}$,
$\bar{\tau}^{1}$ and $\tilde{\tau}^{0}$. Thus the second inequality
of $\eta_{\epsilon}^{{\rm Tr}}$ also holds. The quality constraints
follow by simply tracing out $R$. Finally, the fact that the objective
remains the same follows, again, by starting with \Eqref{purePure}
and applying $\Xi_{Z}^{z}$ and appending $\Pi_{C}^{c}$ as necessary
to conclude that $\tau^{2}=\xi^{2}$. 
\end{proof}
}

\begin{defn}[$\eta$]
 Let $\rho_{0}=\left|{\rm GHZ}\right\rangle \left\langle {\rm GHZ}\right|_{HIJ}$,
$\rho_{1}\in D(ZJ)$ be density matrices and $\Pi_{cz}^{J}\in{\rm Proj}(J)$
be GHZ scenario projectors (Pauli $x$ and Pauli $y$ projectors).
Let $\Pi_{{\rm obj}}$ be as in \Defref{etaLab}. Define 
\[
\eta:=\max\tr(\Pi_{{\rm obj}}\otimes\mathbb{I}_{J}\cdot\rho_{2})
\]
s.t. 
\begin{align*}
\tr_{Z}(\rho_{1}) & =\tr_{IJ}(\rho_{0})\\
\rho_{2} & =\sum_{z,z',c,c'}\left|c\right\rangle \left\langle c'\right|_{C}\otimes(\Pi_{cz}^{J}\otimes\Pi_{z}^{Z}\cdot\rho_{1}\cdot\Pi_{c'z'}^{J}\otimes\Pi_{z'}^{Z}).
\end{align*}
\end{defn}

\begin{lem}
One has for $\epsilon=0$, $\eta_{\epsilon}^{{\rm tr}}=\eta$. 
\end{lem}

\begin{proof}
Suppose that $\tilde{\tau}_{HIJL}^{0}$, $\tau_{HIL}^{0}$, $\tau_{JZ}^{1}$,
$\tilde{\tau}_{JZL}^{1}$ and $\mathcal{P}_{JL}^{cz}$ maximize $\eta_{\epsilon}^{\tr}$
when $\epsilon=0$. This means that $\tau_{HIJ}^{0}=\rho_{HIJ}^{{\rm GHZ}}$
which is a pure state and thus $\tilde{\tau}_{HIJL}^{0}=\tau_{HIJ}^{0}\otimes\tau_{L}$.
Assume, without loss of generality, that $\tau_{L}$ is also a pure
state (if it is not, consider its purification which solves $\eta_{\epsilon}^{\tr}$
with the same value). 

Let $\bar{\tau}_{HIJZL}^{1}$ be such that $\tr_{Z}(\bar{\tau}_{HIJZL}^{1})=\tilde{\tau}_{HIJL}^{0}$
and\footnote{TODO: this may need some justification but should be fine because
$Z$ can be assumed ``classical'' so that something like $\left|00\right\rangle \left\langle 00\right|_{JZ}+\left|11\right\rangle \left\langle 11\right|_{JZ}$
is automatically disallowed. The classicality is enforced by the projector
$\Xi_{{\rm obj}}$.} $\tr_{HI}\bar{\tau}_{HIJZL}^{1}=\tilde{\tau}_{JZL}^{1}$. Now since
$\tilde{\tau}_{HIJL}^{0}$ is a pure state, $\bar{\tau}_{HIJZL}^{1}=\tau_{Z}\otimes\tau_{HIJ}^{0}\otimes\tau_{L}$
for some state $\tau_{Z}$. This structure implies that $\mathcal{I}_{HIZ}\otimes\tr_{L}\mathcal{P}_{JL}^{cz}(\bar{\tau}_{HIJZL}^{1})=\tau_{Z}\otimes\tr_{L}\mathcal{P}_{JL}^{cz}(\tau_{HIZ}^{0}\otimes\tau_{L})=\tau_{Z}\otimes\Xi_{J}^{cz}(\tau_{HIJ}^{0})$.
Consequently, ... {[}Feels strange; what if I did send a coherent
copy of $H$ as $Z$; then my state would have been something else...;
I think I have to argue differently{]}
\end{proof}
%
\begin{proof}
Follows directly from setting $\epsilon=0$ and using \Exaref{replaceM-1}
below to argue (in $\eta_{\epsilon}$) that $P_{cz}$ can be replaced
with $\Pi_{cz}$ when they act on $\tau_{1}$ even though $\epsilon=0$
only yields this result for $\tau_{0}$.
\end{proof}
\begin{example}
\label{exa:replaceM-1}Let $\rho_{AB}$ be a density matrix, $\Pi^{B},\Pi^{\prime B}$
be projectors on $B$ and $M^{B},M^{\prime B}$ be measurement (Kraus)
operators on $B$. Suppose $M^{B}\rho_{AB}M^{\prime B}=\Pi^{B}\rho_{AB}\Pi^{\prime B}$.
Suppose 
\begin{equation}
M^{B}\rho_{AB}M^{\prime B}=\Pi^{B}\rho_{Ab}\Pi^{\prime B}.\label{eq:measureMeasure-1}
\end{equation}
If $\sigma_{AB}$ is another density matrix such that $\tr_{A}(\sigma_{AB})=\tr_{B}(\rho_{AB})$,
then 
\begin{equation}
M^{B}\sigma_{AB}M^{\prime B}=\Pi^{B}\sigma_{AB}\Pi^{\prime B}.\label{eq:replaceMeasurement-1}
\end{equation}
 This follows from Uhlman's theorem which guarantees that there exists
a $U$ acting on system $A$ such that $\left(U\otimes\mathbb{I}_{B}\right)\sigma_{AB}\left(U^{\dagger}\otimes\mathbb{I}_{B}\right)=\rho_{AB}$.
Thus, conjugating \Eqref{measureMeasure-1} with $U\otimes\mathbb{I}_{B}$,
we obtain .
\end{example}


\subsection{The self-testing step {[}Discuss with Tom before writing{]} }
\begin{prop}
For any implementation of the boxes and choice of $\delta>0$, the
joint probability that the test $\Omega$ passes and that the conclusion
$T$ is false is small, i.e. $\Pr[\Omega\cap\bar{T}]\leq\frac{1}{1-\delta+n\delta}\leq\frac{1}{n\delta}$
where the first upper-bound is tight.
\end{prop}


\subsection{Robust Self Testing}
\begin{lem}
Let $a,b,c,x,y,z\in\{0,1\}$. Consider a trio of quantum boxes, specified
by projectors $\{M_{a|x}^{A},M_{b|y}^{B},M_{c|z}^{C}\}$ acting on
finite dimensional Hilbert spaces $\mathcal{H}^{A},\mathcal{H}^{B}$
and $\mathcal{H}^{C}$, and $\left|\psi\right\rangle \in\mathcal{H}^{A}\otimes\mathcal{H}^{B}\otimes\mathcal{H}^{C}=:\mathcal{H}^{ABC}$.
If the trio pass the GHZ test with probability $1-\epsilon$ (for
$1>\epsilon>0$), then there exists a local isometry, 
\[
\Phi=\Phi^{A}\otimes\Phi^{B}\otimes\Phi^{C}:\mathcal{H}^{ABC}\to\mathcal{H}^{ABC}\otimes\mathbb{C}^{2\times3}
\]
and a decreasing function of $\epsilon$, $f(\epsilon)$ such that
\begin{align*}
\left\Vert \Phi\left(\left|\psi\right\rangle \right)-\left|\chi\right\rangle \otimes\left|{\rm junk}\right\rangle \right\Vert  & \le f(\epsilon),\\
\left\Vert \Phi\left(M_{d|t}^{D}\left|\psi\right\rangle \right)-\Pi_{d|t}^{D}\left|{\rm GHZ}\right\rangle \otimes\left|{\rm junk}\right\rangle \right\Vert  & \le f(\epsilon)\quad\forall D\in\{A,B,C\},\text{ and }d,t\in\{0,1\}
\end{align*}
where $\left|{\rm GHZ}\right\rangle =\frac{\left|000\right\rangle +\left|111\right\rangle }{\sqrt{2}}\in\mathbb{C}^{2\times3}$,
$\left|{\rm junk}\right\rangle \in\mathcal{H}^{ABC}$ is some arbitrary
state and $\{\Pi_{a|x}^{A},\Pi_{b|y}^{B},\Pi_{c|z}^{C}\}$ are projectors
corresponding to $\sigma_{x}$ on the first, second and third qubit
of $\left|{\rm GHZ}\right\rangle $ respectively, for $x=0$ and corresponding
to $\sigma_{y}$ for $x=1$, as in \Claimref{Quantum-boxes-pass}.
\end{lem}


\subsection{The continuity argument {[}Enter Jamie{]}}

\bibliographystyle{amsalpha}
\bibliography{DI_WCF_ideas}


\end{document}
