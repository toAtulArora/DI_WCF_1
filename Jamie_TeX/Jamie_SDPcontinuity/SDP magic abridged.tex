\documentclass[11pt]{article}

\usepackage{palatino}
\usepackage{mathpazo}
\usepackage{braket}
\usepackage{amsfonts}
\usepackage{amssymb}
\usepackage{amsmath}
\usepackage{latexsym}
\usepackage{amsthm}
\usepackage[usenames]{color}
\usepackage{hyperref}

\theoremstyle{definition}

\special{papersize=8.5in,11in}
\setlength{\topmargin}{-0.125in}
\setlength{\headheight}{0in}
\setlength{\headsep}{0in}
\setlength{\textheight}{9in}
\setlength{\oddsidemargin}{0in}
\setlength{\textwidth}{6.5in}
\hypersetup{pdfpagemode=UseNone}

\newtheorem{theorem}{Theorem}[section]
\newtheorem{lemma}[theorem]{Lemma}
\newtheorem{question}{Question}
\newtheorem{prop}[theorem]{Proposition}
\newtheorem{cor}[theorem]{Corollary}
\newtheorem{corollary}[theorem]{Corollary}
\theoremstyle{definition}
\newtheorem{definition}[theorem]{Definition}
\newtheorem{remark}[theorem]{Remark}
\newtheorem{fact}[theorem]{Fact}
\newtheorem{example}[theorem]{Example}

\usepackage{color,graphicx}

\newcommand{\complex}{\mathbb{C}} 
\newcommand{\real}{\mathbb{R}} 
%\newcommand{\natural}{\mathbb{N}} 
\newcommand{\rational}{\mathbb{Q}} 

\newcommand{\X}{\mathcal{X}} 
\newcommand{\Y}{\mathcal{Y}} 
\newcommand{\Herm}{\mathrm{Herm}} 

\newcommand{\ip}[2]{\langle #1, #2 \rangle}
\newcommand{\ketbra}[2]{\ket{#1}\bra{#2}} 
\newcommand{\bracket}[2]{\langle #1 | #2 \rangle} 
\newcommand{\kb}[1]{\ket{#1}\bra{#1}} 
\newcommand{\Tr}{\mathrm{Tr}} 
\newcommand{\tr}{\mathrm{Tr}} 

\newcommand{\supp}{\mathrm{supp}}
\newcommand{\F}{\mathcal{F}} 
\newcommand{\I}{\mathbb{1}}

\newcommand{\snote}[1]{\textcolor{magenta}{\textbf{[Jamie: #1]}}}

\begin{document}

%-----------------------------------------------------------------------------%
\title{\bf SDP analysis}
%-----------------------------------------------------------------------------%

\date{\today} 

\author{The Batman, Atul, Tom with the cat} 

\maketitle 

\begin{abstract} 
This note discusses SDP continuity bounds and how they can be used for our analysis. 
\end{abstract}

\tableofcontents 

%%%%%%%%%%%%%%%%%%%%%%%%%%%%%%%%%
%%%%%%%%%%%%%%%%%%%%%%%%%%%%%%%%%   

\section{SDP-valued functions and their continuity} 

A semidefinite program (SDP) is an optimization problem of the form 
\begin{align} 
f(A,B) = \text{maximize:} \quad & \ip{A}{X} \nonumber \\ 
\text{subject to:} \quad 
& \Phi(X) = B \label{SDP} \\ 
& X \geq 0. \nonumber
\end{align}  
We call $f(A,B)$ the value of the semidefinite program which is the supremum of $\ip{A}{X}$ over all $X$ that are feasible ($X \succeq 0$ and $\Phi(X) = B$). 
In this work we wish to view how the value of an SDP changes as you change $A$ and/or $B$. 
%To this end, we define the concept of a support function. 
Ultimately, we wish to know if the value of an SDP is continuous as a function of $A$ and $B$. 
To this end, let us consider the function 
\begin{align} 
h(A) = \text{maximize:} \quad & \ip{A}{X} \\ 
\text{subject to:} \quad 
& X \in C  
\end{align}   
where $C$ is a nonempty, convex set. 
This is a generalization of an SDP which is  convenient for the upcoming analysis.  
Notice that when $C$ is unbounded, it may be the case that $f$ takes the value $+ \infty$. 
Since we cannot count that high, we use the following definition. 

\begin{definition} 
We define the \emph{support} of the function $h$, denoted as $\supp(h)$, as 
\begin{equation} 
\supp(h) := \{ A : h(A) \textup{ is finite} \}.  
\end{equation} 
\end{definition} 

We now show some elementary properties of this function. 

\begin{lemma} 
The support of $h$ is convex and $h$ is a convex function on its support. 
\end{lemma} 

\begin{proof} 
%Since $C$ is convex, we have that $\lambda_1 A_1 + \lambda_2 A_2 \in C$ for all $A_1, A_2 \in C$ and $\lambda_1, \lambda_2 \geq 0$ satisfying $\lambda_1 + \lambda_2 = 1$. 
For $A_1, A_2 \in \supp(h)$ and $\lambda_1, \lambda_2 \geq 0$ satisfying $\lambda_1 + \lambda_2 = 1$, we have  
\begin{align}
h(\lambda_1 A_1 + \lambda_2 A_2) 
& \leq h(\lambda_1 A_1) + h(\lambda_2 A_2) \\ 
& = \lambda_1 h(A_1) + \lambda_2 h(A_2) \\ 
& < + \infty  
\end{align} 
where the last inequality follows from $A_1, A_2 \in \supp(h)$.  
Thus, $\lambda_1 A_1 + \lambda_2 A_2 \in \supp(h)$, proving $\supp(h)$ is a convex set, and $h$ is convex from the above inequalities. 
\end{proof} 

The following corollary follows from the fact that $h$ is convex. 

\begin{corollary} \label{contint}
$h$ is continuous on the interior of its support. 
\end{corollary} 

Another well-known corollary is that $h$ is 
continuous everywhere if $C$ is compact. 
This follows from the above corollary since the support is the entire space. 

\begin{corollary} \label{contint}
If $C$ is compact, $h$ is continuous everywhere. 
\end{corollary} 

%\begin{lemma} \label{compact}
%If $C$ is compact, then $h$ has full support and is continuous everywhere. 
%\end{lemma} 

%\begin{proof} 
%If $C$ is compact, then $h(A) = \ip{A}{X}$ for some $X \in C$ (i.e., $X$ is an optimal solution). 
%Thus, it follows $h$ has full support. 
%Given $A_1, A_2 \in \Herm(\X)$, denote by $X_1, X_2 \in C$ the respective optimal solutions such that ${h(A_1) = \ip{A_1}{X_1}}$ and $h(A_2) = \ip{A_2}{X_2}$. 
%By symmetry, suppose we have ${h(A_1) \geq h(A_2)}$. 
%Therefore,  
%\begin{align} 
%| h(A_1) - h(A_2) | 
%& = h(A_1) - h(A_2) \\ 
%& = \ip{A_1}{X_1} - \ip{A_2}{X_2} \\ 
%& \leq \ip{A_1}{X_1} - \ip{A_2}{X_1} \\ 
%& = \ip{A_1 - A_2}{X_1} \\ 
%& \leq \| A_1 - A_2 \|_2 \cdot \| X_1 \|_2   
%\end{align} 
%where we used the optimality of $X_1$ in the first inequality and Cauchy-Schwarz in the second. 
%Since $X_2 \in C$, which is compact, we know there exists $R > 0$ such that $\| X_1 \|_2 \leq R$. 
%Thus, ${h(A_1) \to h(A_2)}$ as $A_1 \to A_2$, proving $h$ is continuous.  
%\end{proof} 

This may look like it only applies to the case when you change the objective function of an SDP (i.e., the variable $A$), but it often applies to the case when you change the constant in the constraints as well (i.e., the variable $B$). 
This is due to duality theory of semidefinite programming, which we now briefly discuss.   

The dual to the SDP~\eqref{SDP} is given as 
\begin{align} 
g(A,B) = \text{minimize:} \quad & \ip{B}{Y} \\ 
\text{subject to:} \quad 
& \Phi^*(Y) \geq A. 
\end{align}  

The utility of this definition is illustrated in the following fact. 

\begin{fact}[Strong duality] \label{SD}
Under the assumption that $(A,B)$ satisfies $f(A,B) < + \infty$ and there exists $Y$ such that $\Phi^*(Y) > A$, then we have $f(A,B) = g(A,B)$.  
\end{fact} 

\begin{remark} 
One can check that the above analysis for the continuity of the $h$ function is similar when we wish to minimize $\ip{A}{X}$ as well. 
Note that $h$ would be \emph{concave} in this case, but the continuity claims are similar. 
\end{remark}

\end{document} 
